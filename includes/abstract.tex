\cleardoublepage
\thispagestyle{empty}   % keine Kopf-/Fußzeile auf dieser Seite
\pagestyle{empty}       % falls global fancy aktiv ist
\markboth{}{}           % Laufkopf-Markierungen leeren
% abstracts.tex — Deutsch & Englisch untereinander auf EINER Seite
\begin{samepage}
\section*{Kurzreferat}

Diese Arbeit erweitert ein KI-gestütztes Assistenzsystem für das Laserschneiden von Baustahl auf Edelstahl. Ziel ist eine zuverlässige Schätzung der Grathöhe und die Vorbereitung einer belastbaren Rauheitsbewertung. Dazu wurden Schnittabrissgrenzen systematisch bestimmt, neue Bild- und 3D-Daten in der Messzelle aufgenommen und die Messmethodik für Edelstahl angepasst. Ein kompaktes, VGG-ähnliches \ac{CNNs} wurde mit Edelstahldaten neu trainiert. Die Ergebnisse zeigen eine deutlich verbesserte Gratschätzung mit geringerem Dickenbias, besonders bei 10\,mm und 15\,mm. Für die Rauheit ist die Datengrundlage noch zu klein, die Analyse bleibt jedoch konsistent. Die Arbeit schafft die Basis für eine erweiterte Datenerfassung und ein künftiges gemeinsames Modell über Werkstoffe hinweg.

\smallskip

\bigskip

\section*{Abstract}

This thesis extends an AI-assisted cutting assistant from mild steel to stainless steel. The goal is reliable burr prediction and the preparation of a robust roughness assessment. Cut-off limits were determined systematically, new image and 3D data were acquired in the measurement cell, and the measurement procedure was adapted to stainless steel. A compact VGG-like \ac{CNNs} was retrained on stainless data. Results show clearly improved burr prediction with reduced thickness-dependent bias, especially for 10\,mm and 15\,mm sheets. The dataset for roughness modeling is still limited, although the current analysis remains consistent. The work lays the groundwork for extended data collection and for a unified model across materials.

\smallskip
\end{samepage}


\newpage
