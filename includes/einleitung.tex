\chapter{Einführung}

Dieses Kapitel gibt eine Einführung in die Thematik der Projektarbeit. Es werden die Zielsetzung und die geplante Vorgehensweise beschrieben. 

\section{Zielsetzung}
Das Ziel dieser Projektarbeit ist es eine Versuchsplattform für die Automatiseirungstechnik zu entwickeln bei der typische Regelungstechnische Methoden experimentelle untersucht werden und anschließend auf die Zielplattform übertragen werden können. Hierfür soll ein \ac{RCP}-System verwendet werden, um die Regelungsalgorithmen in Echtzeit auf der Zielplattform auszuführen, indem bereits vorhandene Hardware und Software Bausteine genutzt werden. Das \ac{RCP} ist ein Verfahren mit dem zu regelnde Systeme schnell und flexibel entwickelt und getestet werden können. Hierbei ist es nicht notwendig manuelle Implementierung in Programmiersprachen für die Zielhardware zu erstellen, sondern es kann direkt von einer grafischen Simulationsumgebung wie z.B. \textit{Matlab/Simulink} auf die Zielhardware übertragen werden mithilfe von einer automatischen Codegenerierung. Dies ermöglicht eine schnelle Iteration und Anpassung der Regelungsalgorithmen, was besonders in der Entwicklungsphase von Vorteil ist, da somti schnell und kosteneffizient Prototypen erstellt und getestet werden können. Die Versuchsplattform soll einen durchgängigen Prozess von der Modellerstellung über die Simulation bis zur Echtzeitausführung auf der Zielhardware abbilden und dabei die experimentelle Parametrierung, Optimierung sowie die Beobachtung relevanter Signale und Messgrößen ermöglichen, indem eine Kopplung zwischen Entwicklungsrechner und Zielplattform zur Signal und Parameterkommunikation genutzt wird \parencite{HoyosGutierrez2023RCPReview}.

\section{Vorgehensweise}
Die Projekarbeit setzt auf eine Einarbeitung in das Rapid Control Prototyping, um die theoretischen Grundlagen, die verwendeten Begriffe und typische Prozessabläufe einzuordnen und daraus geeignete Vorgehensprinzipien abzuleiten. Aufbauend auf diesem Kenntnisstand werden Anforderungen an die zu entwickelnde Versuchsplattform definiert. Dabei werden funktionale Anforderungen beschrieben, die sich aus den geplanten Experimenten ergeben, sowie nicht funktionale Anforderungen festgelegt, die unter anderem die Umsetzbarkeit, Erweiterbarkeit und Randbedingungen der Nutzung betreffen.

Nachdem die Randbedingungen gesetzt wurden, erfolgt die Auswahl geeigneter Hardware und Software Werkzeuge für die Umsetzung der Versuchsplattform.  Ziel ist es, eine Kombination aus Hardware und Software zu identifizieren, die eine effiziente Entwicklung, Simulation und Echtzeitausführung der Regelungsalgorithmen ermöglicht.

Parallel zum physischen Aufbau wird die Versuchsplattform in Matlab und Simulink modelliert, um ein ausführbares Systemmodell für die Simulation bereitzustellen. In diesem Modell werden Sensorik, Aktorik und die wesentlichen dynamischen Eigenschaften des Prozesses abgebildet, sodass Algorithmen vor der Implementierung auf der realen Plattform unter definierten Bedingungen untersucht werden können. Abschließend wird der Ansatz des Rapid Control Prototyping exemplarisch angewendet, indem ausgewählte Funktionen in Simulink entworfen, simuliert und anschließend mittels automatischer Codegenerierung auf die Zielhardware übertragen werden. Die Implementierung wird getestet und validiert, um die Funktionalität und Leistungsfähigkeit der entwickelten Versuchsplattform zu gewährleisten.

Für die Umsetzung der \ac{RCP}-Methodik wird in der Projektarbeit ein Anwedungssezenario der "Lageschätzung" genutzt. Hierbei soll eine \ac{IMU} simuliert werden und gegebenfalls aus der ausgewählten Hardware eine reale \ac{IMU} ausgelesen werden. Anschließend soll die Lage des Systems in Form von Quaterionen/Eulerwinkeln geschätzt werden (HIER VERWEIS KAPITEL 2.4).Die Lageschätzung soll auf Grundlage einer Sensorfusion funktionieren, indem Beschleunigungs- und Gyroskopdaten kombiniert werden. Hierfür sollen verschiedene Algorithmen implementiert und getestet werden, um die Genauigkeit und Robustheit der Lageschätzung zu bewerten. Ziel ist es, eine zuverlässige Methode zur Bestimmung der Systemlage zu entwickeln, die in Echtzeit auf der \ac{RCP}-Plattform ausgeführt werden kann. 

\section{Anwendungsszenario Lageschätzung}
Nils