% \chapter{Optimierung der Vektorberechnung aus der 3D-Punktwolken}
% \label{chap:3d-punktwolken-optimierung}

% Die mit dem Keyence-System aufgenommene 3D-Punktwolke der Schnittkante bildet die Grundlage für die geometrische Qualitätsauswertung. In der bestehenden Auswertepipeline wird die Punktwolke zunächst segmentiert und in ein lokales Kantenkoordinatensystem überführt. Anschließend erfolgt eine Projektion aus der dreidimensionalen Repräsentation in einen zweidimensionalen Profilverlauf, sodass ein 2D-Vektor entsteht, der den Verlauf des Schneidgrats entlang der Schnittkante beschreibt. Diese Vorgehensweise wurde ursprünglich für Baustahl entwickelt und auf dessen charakteristisch eher wellige, kontinuierliche Gratmorphologie abgestimmt.

% Die nachfolgend beschriebenen Anpassungen betreffen ausschließlich den \textbf{Grat-/Burr}-Verlauf unterteilt in der Ausreißererkennung von Messdaten und der darauffolgenden Interpolation. Analyse ob die Rauheitsanalyse des Baustahl-Modells auch auf Edelstahl erfolgen kann wird in Kapitel ~\ref{sec:roughness-analysis} beschrieben.

% \section{Gratverlaufsanalyse bei Edelstahl}

% Bei Edelstahl zeigt sich eine abweichende, ausgeprägt zackige Gratstruktur mit höheren lokalen Gradienten und diskontinuierlichen Profilabschnitten. Die bislang implementierte Outlier-Korrektur ist konzipiert zur Eliminierung sporadischer Messfehler bei Baustahl. Demnach stuft diese den Gratverlauf von Edelstahl fälschlich als Ausreißer ein und glättet sie übermäßig. Dadurch werden relevante Merkmale des Edelstahlgrats unterdrückt und der resultierende Vektorverlauf in Richtung eines künstlich „glatten“ Profils verzerrt. Den unterschiedlichen Gratverlauf ist in den obigen Handscanneraufnahmen von Baustahl und Edelstahl ersichtlich (siehe Abbildungen ~\ref{fig:burr_baustahl} und ~\ref{fig:burr_edelstahl}).

% Erschwerend kommt hinzu, dass im Messprozess partiell überbeschattete Bereiche auftreten können, die vom Sensor nicht erfasst werden. In der bisherigen Pipeline werden solche Lücken durch Interpolation geschlossen, deren Parameter auf die kontinuierlichen Profile von Baustahl zugeschnitten sind. Für den zackigen Edelstahlgrat führt dies zu einer zu starken Annäherung an glatte Zwischenverläufe und damit zu einem Verlust an formcharakteristischer Information.

% Zur materialspezifischen Anpassung werden daher zwei Kernmodule überarbeitet. Die Ausreißererkennung \cref{sec:outlier-correction} mit nachgelagerter Korrektur und die Interpolation \cref{sec:Interpolation} fehlender Stützstellen. In der Ausreißererkennung werden die Schwellwerte und die zugrunde liegenden Sensitivitätsmaße an die höhere lokale Krümmung und den gesteigerten Kantenkontrast des Edelstahlgrats angepasst. Ziel ist eine Differenzierung zwischen echten Messfehlern und materialtypischen Hochfrequenzanteilen. Entsprechend werden Glättungsschritte zurückgenommen, sodass signifikante Gratflanken erhalten bleiben. 

% Für die Interpolation wird ein konservativer Ansatz gewählt, der Lückenschlüsse bevorzugt entlang lokal konsistenter Nachbarschaften vornimmt und globale, stark glättende Approximationen vermeidet. Damit wird erreicht, dass der rekonstruierte 2D-Vektor fehlende Messpunkte plausibel ergänzt, ohne die charakteristische Zackigkeit des Edelstahlgrats zu nivellieren.

% \subsection{Optimierung der Ausreißererkennung}
% \label{sec:outlier-correction}

% Ziel der Ausreißerbehandlung im \textbf{Burr}-Verlauf ist die zuverlässige Trennung zwischen echten Messfehlern und materialtypischen Hochfrequenzanteilen des Edelstahlgrats. Die Verarbeitung ist zweistufig aufgebaut. Zuerst erfolgt ein Vergleich des gemessenen Profils mit einem lokal geglätteten Referenzsignal zur Detektion potenzieller Ausreißerpunkte. Anschließend werden entstandene Lücken kontrolliert rekonstruiert. Der zugehörige Pseudocode ist in \cref{alg:simple-burr} dargestellt.

% In \texttt{outlier\_correction\_burr} dient ein gleitender Mittelwert auf \texttt{z\_vec} als Referenzsignal. Die punktweise Abweichung \(\Delta=\lvert z-\overline{z}_{\text{MA}}\rvert\) wird mit dem Schwellwert \texttt{threshold} verglichen. Alle Punkte oberhalb dieses Schwellwertes bilden die Ausreißermaske. Überschreitet der Anteil markierter Punkte den Grenzwert \texttt{max\_nan\_values\_perc}, wird der Lauf verworfen und \texttt{None} zurückgegeben. Dieses Abbruchkriterium wirkt als Qualitätswächter bei stark gestörten Profilen. Liegt der Anteil darunter, werden die Ausreißer in \texttt{z\_vec} zu \texttt{NaN} gesetzt und mit \texttt{smooth\_nan\_values} rekonstruiert. Die Rekonstruktion schließt kurze Lücken entlang der lokalen Nachbarschaft und erhält charakteristische Gratflanken, sodass der resultierende Vektor die materialspezifische Zackigkeit beibehält.

% Die Parametrisierung orientiert sich an der spektralen Struktur des Profils und an der Messrauschamplitude. Das Fenster \texttt{window} bestimmt die lokale Glättung des Referenzsignals. Ein kleineres Fenster reagiert sensibler auf kurzwellige Änderungen, während ein größeres Fenster stärker glättet. Der Schwellwert \texttt{threshold} definiert die Minimalhöhe einer Abweichung, ab der ein Punkt als Ausreißer gilt. Für Edelstahl wird ein höherer Schwellwert gewählt, um die ausgeprägten kurzwelligen Strukturen nicht fälschlich zu maskieren. Der Grenzwert \texttt{max\_nan\_values\_perc} begrenzt die zulässige Ausreißerquote und verhindert Rekonstruktionen auf unzureichender Datenbasis. Die konkreten Werte für Edelstahl und die Referenzkonfiguration für Baustahl sind in \cref{lst:params-outlier-stainless} und \cref{lst:params-outlier-mild} angegeben.

% Aus methodischer Sicht besitzt die Vorgehensweise eine lineare Laufzeit in der Profil­länge und ist damit für große Punktzahlen geeignet. Die Nutzung eines zentrierten gleitenden Mittelwerts sorgt für eine symmetrische Referenzbildung. Randbereiche werden durch die Faltung im Same-Modus in der Länge konsistent gehalten. Längere Messausfälle werden bewusst nicht vollständig interpoliert, da dies die Formtreue des Burr-Verlaufs beeinträchtigen würde. In solchen Fällen greift das Abbruchkriterium und markiert die entsprechende Profilzeile für eine erneute Datenerfassung oder für eine angepasste Parametrisierung.

% \begin{algorithm}[H]
% \DontPrintSemicolon
% \caption{Einfache Ausreißer-Korrektur im Burr-Profil \,(vgl. Parameter in \cref{lst:params-outlier-stainless,lst:params-outlier-mild})}
% \label{alg:simple-burr}
% \KwIn{Positionen $x$, Höhen $z$, Schwelle $\tau$, Fenster $w$, max.~Anteil $\alpha$}
% \KwOut{$(x,\ z_{\text{clean}})$ oder \texttt{None}}

% % 1) Referenzkurve
% ref $\leftarrow$ MOVING\_AVERAGE($z$, $w$, centered)\;   % gleitender Mittelwert

% % 2) Abweichung und Maske
% diff $\leftarrow$ ABS($z$ $-$ ref)\;
% mask $\leftarrow$ (diff $>$ $\tau$)\;
% anteil $\leftarrow$ COUNT(mask) / LENGTH($z$)\;

% % 3) Abbruch bei zu vielen Ausreißern
% \If{anteil $>$ $\alpha$}{\Return (\texttt{None}, \texttt{None})}

% % 4) Ausreißer entfernen und Lücken glätten
% set $z$[mask] $\leftarrow$ NaN\;
% $z_{\text{clean}}$ $\leftarrow$ SMOOTH\_NAN\_VALUES($x$, $z$)\; % kurze Lücken füllen

% \Return ($x$, $z_{\text{clean}}$)
% \end{algorithm}

% In der Edelstahl-Pipeline wurden der Schwellwert der Burr-Ausreißererkennung angehoben und das Fenster leicht verkürzt. Ziel ist die Erhaltung materialspezifischer Hochfrequenzanteile bei zugleich robuster Erkennung echter Messfehler. Die verwendeten Parameter sind in \cref{lst:params-outlier-stainless} dokumentiert. Zum Vergleich ist die bisherige Baustahl-Konfiguration in \cref{lst:params-outlier-mild} angegeben. Für die Rauheitsanalyse wurden keine Parameter geändert.

% \begin{lstlisting}[caption={Pipeline-Parameter Outlier Correction (Edelstahl, Burr-Verlauf) \,(zu \cref{alg:simple-burr})}, label={lst:params-outlier-stainless}]
% # Parameter burr outlier correction (Edelstahl)
% burr_outlier_threshold : 0.06  # Threshold for Moving Average Difference Filter [mm]
% burr_outlier_window    : 9     # Window for Moving Average Difference Filter [samples]
% \end{lstlisting}

% \begin{lstlisting}[caption={Pipeline-Parameter Outlier Correction (Baustahl, Burr-Verlauf) \,(zu \cref{alg:simple-burr})}, label={lst:params-outlier-mild}]
% # Parameter burr outlier correction (Baustahl)
% burr_outlier_threshold : 0.03  # Threshold for Moving Average Difference Filter [mm]
% burr_outlier_window    : 10    # Window for Moving Average Difference Filter [samples]
% \end{lstlisting}

% \subsection{Optimierung der Interpolation}
% \label{sec:Interpolation}

% Die Interpolation rekonstruiert fehlende Messwerte (\texttt{NaN}), die durch Ausreißerkennzeichnung oder unvollständige Erfassung entstehen. Ziel ist die Wiederherstellung eines plausiblen Gratverlaufs, daher werden nur kurze, lokal begrenzte Lücken gefüllt und größere Ausfälle bleiben markiert. Durch die angepasste Ausreißererkennung werden weniger echte Messpunkte fälschlich als Ausreißer markiert, sodass deutlich weniger Lücken entstehen und nicht mehr unnötig interpoliert wird. Am Interpolationsskript wurden keine Änderungen vorgenommen. Alle Konstanten bleiben unverändert, insbesondere die maximale Lückenlänge sowie Glättungs- und Fensterparameter. Der zugehörige Quellcode ist im Anhang dokumentiert. Ein veranschaulichter Gratverlauf ist im folgendem Kapitel ~\ref{sec:validierung-burr} in Abbildung ~\ref{fig:burr-compare} dargestellt.

% \section{Rauheitsanalyse bei Edelstahl}
% \label{sec:roughness-analysis}

% Für die Rauheitsanalyse wird die bestehende Auswertelogik aus der Baustahl-Pipeline übernommen und auf Edelstahl angewendet. Ziel ist die Überprüfung, ob die Parameter und Rechenschritte ohne Anpassungen auch für Edelstahl robuste und nachvollziehbare Werte liefern. Dabei liegt der Fokus auf der Erkennung unplausibler Ergebnisse, die auf ungeeignete Aufnahmebedingungen oder auf notwendigen Anpassungsbedarf im Skript hinweisen könnten.

% Die Bildaufnahmeeinstellungen des Handscanners wurden zuvor für Edelstahl optimiert, um Effekte durch erhöhte Reflexion zu minimieren, siehe \cref{chap:handscanner-setup}. Auftretende Unplausibilitäten wären damit primär dem Auswertungsskript zuzuordnen. Entsprechend wurde die Rauheitsanalyse anhand einer Stichprobe validiert. Untersucht wurden zehn Edelstahlbleche mit jeweils vier Schnittkanten. Die Prüfung umfasste Konsistenztests über die gesamte Kantenlänge, die Kontrolle auf Ausreißer sowie die Sichtprüfung auf Sättigungseffekte und Messlücken. Keine der vierzig untersuchten Kanten zeigte auffällig abweichende Rauheitswerte. Das Skript zur Rauheitsanalyse wird daher unverändert beibehalten.

% Die Auswertung erfolgt abschnittsweise entlang der Schnittkante. Die Kante wird von oben nach unten in Segmente der Länge \(0{,}3\,\mathrm{mm}\) unterteilt. Die X-Achse gibt die Position entlang der Schnittkante \(x\,[\mathrm{mm}]\) an, die Y-Achse den für eine Segmentlänge von \(0{,}3\,\mathrm{mm}\) berechneten Rauheitswert \(R_a\,[\mu\mathrm{m}]\).Die Skalierung der x-Achse auf den Bereich \(2{,}1\,\mathrm{cm}\) bis \(7{,}6\,\mathrm{cm}\) ergibt sich aus der Blechbreite von \(10\,\mathrm{cm}\). Ausgewertet wird bewusst nur der mittlere Abschnitt, um beschleunigungsbedingte Einflüsse des Laserschneidprozesses an Ein- und Ausfahrt des Schneidkopfes zu vermeiden. Für jedes Segment wird die Rauheitskennzahl \(R_a\) berechnet und als Profil über der Kantenlänge dargestellt. Die Abbildung ~\ref{fig:rauheitsanalyse} zeigt ein typisches Ergebnis. Das Profil weist eine geringe Streuung um den Mittelwert auf und enthält keine systematischen Trends oder Ausreißer. Dies bestätigt die Eignung der bestehenden Parametrisierung für Edelstahl.

% \begin{figure}[h]
%   \centering
%   \includegraphics[width=.6\linewidth]{rauheitsanalyse.png}%
%   \caption{Beispielhafter Verlauf der Rauheitskennzahl \(R_a\) entlang einer Schnittkante im ersten Abschnitt mit der Breite 0,3 mm.}
%   \label{fig:rauheitsanalyse}
% \end{figure}


% \section {Validierung der Messoptimierungen}
% \label{sec:validierung-burr}

% Im folgendem Kapitel werden die verbesserten Messmethoden für die Grat- und Rauheitsanalyse validiert, um bestenfalls die Optimierung für Datenerfassung zu nutzen, so dass das KI-Modell neu trainiert werden kann. 

% \subsection{Validierung der verbesserten Ausreißererkennung im Burr-Verlauf}

% In diesem Abschnitt werden die edelstahl-optimierten Einstellungen der Ausreißererkennung validiert. Verglichen wird die frühere Baustahl-Parametrierung mit der angepassten Erkennung für Edelstahl.

% Die Abbildung~\ref{fig:burr-compare} zeigt den Vergleich der verbsserten und der unverbeserten Ausreißererkennung. Hierfür wurden jeweils das gleiche Sektionsbild des gleichen Edelstahlblechs genutzt. Die rote Linie ist der abgeleitete Burr-Verlauf. Auf dem linken Schaubild ist die unverbesserte Ausreißererkennung erkennbar, welche für Baustahl werdet wird und rechts die verbesserte Ausreißererkennung. Demnach ist der verbesserte Verlauf durchgängig, es entstehen weniger fehlerhafte Lücken und die interpolierten Abschnitte sind plausibler. Quantitativ sinkt der Burr-Wert der gezeigten Sektion von \(681{,}99\,\mu\mathrm{m}\) (Baustahl-Parameter) auf \(650{,}88\,\mu\mathrm{m}\) (Edelstahl-Parameter), jedoch erfolgt die einschätzende Bewertung visuell. Es wird geprüft, wie gut die rote Linie dem sichtbaren Gratverlauf folgt.  
% Demnach ist die verbesserte Ausreißererkennung für Edelstahl validiert.
% \begin{figure}[htbp]
%   \centering
%   \begin{minipage}[t]{0.48\linewidth}
%     \centering
%     \includegraphics[width=\linewidth]{outlier_detection_baustahl.png}
%     \vspace{0.3em}
%     {\small\emph{Unverbesserte Ausreißererkennung: sichtbare Lücken und Fehlverfolgungen.}}
%   \end{minipage}\hfill
%   \begin{minipage}[t]{0.48\linewidth}
%     \centering
%     \includegraphics[width=\linewidth]{outlier_detection_edelstahl.png}
%     \vspace{0.3em}
%     {\small\emph{Verbesserte Ausreißererkennung: durchgängiger Verlauf, plausiblere Interpolation.}}
%   \end{minipage}
%   \caption[Ausreißererkennung im Burr-Verlauf]{Links ist die unverbesserte Ausreißererkennung erkennbar, welche für Baustahl werdet wird und rechts die verbesserte Ausreißererkennung. Die rote Kurve ist der abgeleitete Burr-Verlauf.}
%   \label{fig:burr-compare}
% \end{figure}

% \subsection{Validierung der bestehenden Rauheitsanalyse an den neuen Edelsdtahldaten}
% In diesem Abschnitt wird die Rauheitsanalyse validiert. Die Berechnungsmethode bleibt unverändert. Die Belichtung des Handscanners wurde so angepasst, dass die Struktur der Schnittfläche zuverlässig erfasst wird (siehe Abschnitt~\ref{chap:handscanner-setup}). Bei der Analyse in Kapitel ~\ref{fig:rauheitsanalyse} traten keine systematischen Ausreißer auf, daher wird die Berechnung vorerst unverändert weitergeführt. Anpassungen werden erst nach weiterführenden Modelltrainings geprüft.
