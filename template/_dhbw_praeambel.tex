%%% Präambel %%%
% Sprachumschaltung DE/EN
\usepackage{ifthen}
\newcommand{\DEoEN}[2]{\ifthenelse{\equal{\meineSprache}{DE}}{#1}{#2}}

% Zeichencodierung/Fonts
\usepackage[utf8]{inputenc}
\usepackage[T1]{fontenc}

% Farben + Code-Listings
\usepackage{xcolor}      % statt 'color' (mehr Features)
\usepackage{listings}
% Grund-Setup für Listings (Umlaute/EUR etc.)
\lstset{numbers=left, numberstyle=\tiny, numbersep=5pt, texcl=true}
\lstset{literate=
{Ö}{{\"O}}1 {Ä}{{\"A}}1 {Ü}{{\"U}}1 {ß}{{\ss}}2
{ü}{{\"u}}1 {ä}{{\"a}}1 {ö}{{\"o}}1 {€}{{\euro}}1
}
% Lesbares Listing-Layout + automatischer Zeilenumbruch
\lstdefinestyle{code}{
  numbers=left, numberstyle=\tiny, numbersep=5pt,
  breaklines=true, breakatwhitespace=true,
  columns=fullflexible, keepspaces=true, tabsize=2,
  basicstyle=\ttfamily\small,
  postbreak=\mbox{\textcolor{gray}{$\hookrightarrow$}}\space
}
\lstset{style=code}

% Seitenränder
\usepackage[
  left=2.5cm, right=2.5cm,
  top=2.5cm, bottom=2.5cm,
  foot=12mm, includefoot
]{geometry}

% Sprache + Anführungszeichen
\DEoEN{
  \usepackage[ngerman]{babel}
  \usepackage[babel,german=quotes]{csquotes}
}{
  \usepackage[english]{babel}
  \usepackage[babel,english=british]{csquotes}
}

% Listen, Grafiken, Zeilenabstand
\usepackage{enumerate}
\usepackage{graphicx}
\graphicspath{{img/}}
\usepackage[onehalfspacing]{setspace}

\usepackage[ruled,vlined]{algorithm2e}
\SetKwInput{KwIn}{Input}
\SetKwInput{KwOut}{Output}
\SetKw{KwInit}{initialize}
\SetKw{KwRet}{return}


% Testtext, Akronyme
\usepackage{blindtext}
% \usepackage{color}  % durch xcolor ersetzt
\usepackage[nohyperlinks]{acronym}

% Literatur (biblatex + DHBW-Config)
\usepackage[
  backend=biber,
  bibstyle=_dhbw_authoryear,maxbibnames=99,
  citestyle=authoryear, dashed=false,
  uniquename=true, useprefix=true,
  bibencoding=utf8
]{biblatex}
\input{template/_dhbw_biblatex-config.tex}

% Verzeichnisse/Tabellen
\usepackage{booktabs}
\usepackage{tabularx}
\usepackage{ragged2e} % für \RaggedRight in X-Spalten
\usepackage{float}

\newcolumntype{Y}{>{\RaggedRight\arraybackslash}X}
\usepackage{tocloft}
\usepackage{multirow}
\usepackage{amsmath}
\usepackage{amssymb}
\usepackage{booktabs}


% Hyperlinks
\usepackage[hypertexnames=false]{hyperref}

% --- Quellcode statt Listing überall ---
\usepackage[capitalise,nameinlink,noabbrev]{cleveref}

% Bezeichnungen des listings-Pakets
\usepackage{listings}
\renewcommand{\lstlistingname}{Quellcode}
\renewcommand{\lstlistlistingname}{Quellcodeverzeichnis}

% Namen für \autoref (hyperref)
\providecommand*{\listingautorefname}{Quellcode}
\providecommand*{\lstlistingautorefname}{Quellcode}

% cleveref: beide möglichen Typen abdecken
\crefname{listing}{Quellcode}{Quellcodes}
\Crefname{listing}{Quellcode}{Quellcodes}
\crefname{lstlisting}{Quellcode}{Quellcodes}
\Crefname{lstlisting}{Quellcode}{Quellcodes}

% Bessere Umbrüche in \url{…}
\usepackage{xurl}
\urlstyle{tt}

% Anhangszähler/Verzeichnis (wie in deiner Vorlage)
\newcounter{anhcnt}\setcounter{anhcnt}{0}
\newlistof{anhang}{app}{}
\newcommand{\anhang}[1]{%
  \refstepcounter{anhcnt}\setcounter{anhteilcnt}{0}
  \section*{\appendixname\ \theanhcnt: #1}
  \addcontentsline{app}{section}{\protect\numberline{\appendixname\ \theanhcnt}#1}\par
}
\newcounter{anhteilcnt}\setcounter{anhteilcnt}{0}
\newcommand{\anhangteil}[1]{%
  \refstepcounter{anhteilcnt}
  \subsection*{\appendixname\ \arabic{anhcnt}/\arabic{anhteilcnt}: #1}
  \addcontentsline{app}{subsection}{\protect\numberline{\appendixname\ \theanhcnt/\arabic{anhteilcnt}}#1}\par
}
\renewcommand{\theanhteilcnt}{\appendixname\ \theanhcnt/\arabic{anhteilcnt}}

% tocloft-Einrückungen für Anhangverzeichnis
\makeatletter
\newcommand{\abstaendeanhangverzeichnis}{%
  \renewcommand*{\l@section}{\@dottedtocline{1}{0em}{5.5em}}
  \renewcommand*{\l@subsection}{\@dottedtocline{2}{2.3em}{6.5em}}
}
% Einträge LOF/LOT und Quellcodeverzeichnis (LOL) angeglichen
\renewcommand*{\l@figure}{\@dottedtocline{1}{0em}{2.3em}}
\renewcommand*{\l@table}{\@dottedtocline{1}{0em}{2.3em}}
\renewcommand*{\l@lstlisting}{\@dottedtocline{1}{0em}{2.3em}}
\makeatother

% Fortlaufende Zähler über Kapitel hinweg
\usepackage{chngcntr}
\counterwithout{figure}{chapter}
\counterwithout{table}{chapter}
\counterwithout{footnote}{chapter}

% Kopfzeilen (KOMA-Script)
\usepackage[automark]{scrlayer-scrpage}
\input{template/_dhbw_kopfzeilen.tex}

% Euro-Zeichen
\usepackage{textcomp}
\usepackage{eurosym}
\renewcommand{\texteuro}{\euro}  % ACHTUNG: nach hyperref laden!

% Kompatibilität KOMA-Script
\usepackage{scrhack}

% Abstände bei Kapitelüberschriften (inkl. Verzeichnisse)
\renewcommand*\chapterheadstartvskip{\vspace*{-\topskip}}
\newcommand{\myBeforeTitleSkip}{1mm}
\newcommand{\myAfterTitleSkip}{10mm}
\setlength\cftbeforetoctitleskip{\myBeforeTitleSkip}
\setlength\cftbeforeloftitleskip{\myBeforeTitleSkip}
\setlength\cftbeforelottitleskip{\myBeforeTitleSkip}
\setlength\cftaftertoctitleskip{\myAfterTitleSkip}
\setlength\cftafterloftitleskip{\myAfterTitleSkip}
\setlength\cftafterlottitleskip{\myAfterTitleSkip}

% Anhang beginnen
\newcommand{\startAnhang}{%
  \chapter*{\appendixname}
  \addcontentsline{toc}{chapter}{\appendixname}
  \section*{\anhangVzBezeichnung}
  \vspace{-8em}
  % vor \listofanhang müssen Einrückungen angepasst werden
  \abstaendeanhangverzeichnis
  \spezialkopfzeile{\DEoEN{Anhang}{Appendix}}
}

% Abkürzungsverzeichnis beginnen
\newcommand{\startAbkVerzeichnis}{%
  \chapter*{\abkVzBezeichnung}
  \addcontentsline{toc}{chapter}{\abkVzBezeichnung}
}

% Zeilenabstand in Tabellen schnell ändern
\newcommand{\ra}[1]{\renewcommand{\arraystretch}{#1}}

% Sprachspezifische Überschriften
\DEoEN{%
  \newcommand{\abkVzBezeichnung}{Abkürzungsverzeichnis}
  \newcommand{\anhangVzBezeichnung}{Anhangverzeichnis}
  \renewcaptionname{ngerman}{\refname}{Literaturverzeichnis}
  \renewcaptionname{ngerman}{\figurename}{Abb.}
  \renewcaptionname{ngerman}{\tablename}{Tab.}
}{%
  \newcommand{\abkVzBezeichnung}{Abbreviations}
  \newcommand{\anhangVzBezeichnung}{Appendix directory}
  \renewcaptionname{english}{\contentsname}{Table of Contents}
  \renewcaptionname{english}{\figurename}{Fig.}
  \renewcaptionname{english}{\tablename}{Tab.}
}

%%% Ende der Präambel %%%
