%%% LaTeX-Vorlage Version 2.1 %%%

% TODO: individuelle Einstellungen (Name, Titel etc.)
% -> bitte in Konfigurationsdatei anpassen
\input{includes/config.tex}


% Grundlegende Dokumenteneigenschaften gemäß DHBW-Vorgaben
\documentclass[a4paper,fontsize=11pt,oneside,parskip=half,headings=normal,listof=nochaptergap]{scrreprt} 
% \usepackage{showframe} % nur für Kontrolle der Ränder 

%%% Präambel einbinden (mit Festlegungen gemäß DHBW-Vorgaben) %%%
%%% Präambel %%%
% Sprachumschaltung DE/EN
\usepackage{ifthen}
\newcommand{\DEoEN}[2]{\ifthenelse{\equal{\meineSprache}{DE}}{#1}{#2}}

% Zeichencodierung/Fonts
\usepackage[utf8]{inputenc}
\usepackage[T1]{fontenc}

% Farben + Code-Listings
\usepackage{xcolor}      % statt 'color' (mehr Features)
\usepackage{listings}
% Grund-Setup für Listings (Umlaute/EUR etc.)
\lstset{numbers=left, numberstyle=\tiny, numbersep=5pt, texcl=true}
\lstset{literate=
{Ö}{{\"O}}1 {Ä}{{\"A}}1 {Ü}{{\"U}}1 {ß}{{\ss}}2
{ü}{{\"u}}1 {ä}{{\"a}}1 {ö}{{\"o}}1 {€}{{\euro}}1
}
% Lesbares Listing-Layout + automatischer Zeilenumbruch
\lstdefinestyle{code}{
  numbers=left, numberstyle=\tiny, numbersep=5pt,
  breaklines=true, breakatwhitespace=true,
  columns=fullflexible, keepspaces=true, tabsize=2,
  basicstyle=\ttfamily\small,
  postbreak=\mbox{\textcolor{gray}{$\hookrightarrow$}}\space
}
\lstset{style=code}

% Seitenränder
\usepackage[
  left=2.5cm, right=2.5cm,
  top=2.5cm, bottom=2.5cm,
  foot=12mm, includefoot
]{geometry}

% Sprache + Anführungszeichen
\DEoEN{
  \usepackage[ngerman]{babel}
  \usepackage[babel,german=quotes]{csquotes}
}{
  \usepackage[english]{babel}
  \usepackage[babel,english=british]{csquotes}
}

% Listen, Grafiken, Zeilenabstand
\usepackage{enumerate}
\usepackage{graphicx}
\graphicspath{{img/}}
\usepackage[onehalfspacing]{setspace}

\usepackage[ruled,vlined]{algorithm2e}
\SetKwInput{KwIn}{Input}
\SetKwInput{KwOut}{Output}
\SetKw{KwInit}{initialize}
\SetKw{KwRet}{return}


% Testtext, Akronyme
\usepackage{blindtext}
% \usepackage{color}  % durch xcolor ersetzt
\usepackage[nohyperlinks]{acronym}

% Literatur (biblatex + DHBW-Config)
\usepackage[
  backend=biber,
  bibstyle=_dhbw_authoryear,maxbibnames=99,
  citestyle=authoryear, dashed=false,
  uniquename=true, useprefix=true,
  bibencoding=utf8
]{biblatex}
\input{template/_dhbw_biblatex-config.tex}

% Verzeichnisse/Tabellen
\usepackage{booktabs}
\usepackage{tabularx}
\usepackage{ragged2e} % für \RaggedRight in X-Spalten
\usepackage{float}

\newcolumntype{Y}{>{\RaggedRight\arraybackslash}X}
\usepackage{tocloft}
\usepackage{multirow}
\usepackage{amsmath}
\usepackage{amssymb}
\usepackage{booktabs}


% Hyperlinks
\usepackage[hypertexnames=false]{hyperref}

% --- Quellcode statt Listing überall ---
\usepackage[capitalise,nameinlink,noabbrev]{cleveref}

% Bezeichnungen des listings-Pakets
\usepackage{listings}
\renewcommand{\lstlistingname}{Quellcode}
\renewcommand{\lstlistlistingname}{Quellcodeverzeichnis}

% Namen für \autoref (hyperref)
\providecommand*{\listingautorefname}{Quellcode}
\providecommand*{\lstlistingautorefname}{Quellcode}

% cleveref: beide möglichen Typen abdecken
\crefname{listing}{Quellcode}{Quellcodes}
\Crefname{listing}{Quellcode}{Quellcodes}
\crefname{lstlisting}{Quellcode}{Quellcodes}
\Crefname{lstlisting}{Quellcode}{Quellcodes}

% Bessere Umbrüche in \url{…}
\usepackage{xurl}
\urlstyle{tt}

% Anhangszähler/Verzeichnis (wie in deiner Vorlage)
\newcounter{anhcnt}\setcounter{anhcnt}{0}
\newlistof{anhang}{app}{}
\newcommand{\anhang}[1]{%
  \refstepcounter{anhcnt}\setcounter{anhteilcnt}{0}
  \section*{\appendixname\ \theanhcnt: #1}
  \addcontentsline{app}{section}{\protect\numberline{\appendixname\ \theanhcnt}#1}\par
}
\newcounter{anhteilcnt}\setcounter{anhteilcnt}{0}
\newcommand{\anhangteil}[1]{%
  \refstepcounter{anhteilcnt}
  \subsection*{\appendixname\ \arabic{anhcnt}/\arabic{anhteilcnt}: #1}
  \addcontentsline{app}{subsection}{\protect\numberline{\appendixname\ \theanhcnt/\arabic{anhteilcnt}}#1}\par
}
\renewcommand{\theanhteilcnt}{\appendixname\ \theanhcnt/\arabic{anhteilcnt}}

% tocloft-Einrückungen für Anhangverzeichnis
\makeatletter
\newcommand{\abstaendeanhangverzeichnis}{%
  \renewcommand*{\l@section}{\@dottedtocline{1}{0em}{5.5em}}
  \renewcommand*{\l@subsection}{\@dottedtocline{2}{2.3em}{6.5em}}
}
% Einträge LOF/LOT und Quellcodeverzeichnis (LOL) angeglichen
\renewcommand*{\l@figure}{\@dottedtocline{1}{0em}{2.3em}}
\renewcommand*{\l@table}{\@dottedtocline{1}{0em}{2.3em}}
\renewcommand*{\l@lstlisting}{\@dottedtocline{1}{0em}{2.3em}}
\makeatother

% Fortlaufende Zähler über Kapitel hinweg
\usepackage{chngcntr}
\counterwithout{figure}{chapter}
\counterwithout{table}{chapter}
\counterwithout{footnote}{chapter}

% Kopfzeilen (KOMA-Script)
\usepackage[automark]{scrlayer-scrpage}
\input{template/_dhbw_kopfzeilen.tex}

% Euro-Zeichen
\usepackage{textcomp}
\usepackage{eurosym}
\renewcommand{\texteuro}{\euro}  % ACHTUNG: nach hyperref laden!

% Kompatibilität KOMA-Script
\usepackage{scrhack}

% Abstände bei Kapitelüberschriften (inkl. Verzeichnisse)
\renewcommand*\chapterheadstartvskip{\vspace*{-\topskip}}
\newcommand{\myBeforeTitleSkip}{1mm}
\newcommand{\myAfterTitleSkip}{10mm}
\setlength\cftbeforetoctitleskip{\myBeforeTitleSkip}
\setlength\cftbeforeloftitleskip{\myBeforeTitleSkip}
\setlength\cftbeforelottitleskip{\myBeforeTitleSkip}
\setlength\cftaftertoctitleskip{\myAfterTitleSkip}
\setlength\cftafterloftitleskip{\myAfterTitleSkip}
\setlength\cftafterlottitleskip{\myAfterTitleSkip}

% Anhang beginnen
\newcommand{\startAnhang}{%
  \chapter*{\appendixname}
  \addcontentsline{toc}{chapter}{\appendixname}
  \section*{\anhangVzBezeichnung}
  \vspace{-8em}
  % vor \listofanhang müssen Einrückungen angepasst werden
  \abstaendeanhangverzeichnis
  \spezialkopfzeile{\DEoEN{Anhang}{Appendix}}
}

% Abkürzungsverzeichnis beginnen
\newcommand{\startAbkVerzeichnis}{%
  \chapter*{\abkVzBezeichnung}
  \addcontentsline{toc}{chapter}{\abkVzBezeichnung}
}

% Zeilenabstand in Tabellen schnell ändern
\newcommand{\ra}[1]{\renewcommand{\arraystretch}{#1}}

% Sprachspezifische Überschriften
\DEoEN{%
  \newcommand{\abkVzBezeichnung}{Abkürzungsverzeichnis}
  \newcommand{\anhangVzBezeichnung}{Anhangverzeichnis}
  \renewcaptionname{ngerman}{\refname}{Literaturverzeichnis}
  \renewcaptionname{ngerman}{\figurename}{Abb.}
  \renewcaptionname{ngerman}{\tablename}{Tab.}
}{%
  \newcommand{\abkVzBezeichnung}{Abbreviations}
  \newcommand{\anhangVzBezeichnung}{Appendix directory}
  \renewcaptionname{english}{\contentsname}{Table of Contents}
  \renewcaptionname{english}{\figurename}{Fig.}
  \renewcaptionname{english}{\tablename}{Tab.}
}

%%% Ende der Präambel %%%


%%% Name der eigenen Literatur-Datenbank (ggf. anpassen) %%%
\bibliography{includes/literatur-datenbank.bib}

\begin{document}
%%% Deckblatt gemäß DHBW-Vorgaben einbinden (keine Anpassung nötig) %%% 
    \begin{titlepage}
        \centering % Zentriert den gesamten Inhalt auf der Seite
       
        % Logos
        \includegraphics[height=60pt]{TruLogo_Print_RGB.jpg}\hfill\includegraphics[height=60pt]{DHBW_Logo.jpg}
       
        \vspace{1.5cm} % Abstand nach Logos
       
        % Titel
        {\LARGE\textsc{Erweiterung eines KI-gestützten Assistenzsystems zur Optimierung von
Laserschneidparametern für Edelstahlbleche}}\\[1.5cm]
       
        {\Large\textbf{Projektarbeit T2000}}\\[1cm]
       
        % Studiengang und Hochschule
        {\large
            des Studienganges Elektrotechnik\\
            Fachrichtung Automation\\
            an der Dualen Hochschule Baden-Württemberg\\
            Standort Stuttgart
        }\\[1.5cm]
       
        % Name des Autors
        {\large\textbf{Tudor Lupsa}}\\[0.5cm]
       
        % Abgabedatum
        {\large 08.09.2025}\\[2cm]
       
        % Weitere Informationen
        \begin{tabular}{ll}
            \textbf{Bearbeitungszeitraum} & 02.06.25 - 08.09.25\\
            \textbf{Matrikelnummer, Kurs} & 1491114, TEL23GR3\\
            \textbf{Dualer Partner} & TRUMPF SE+Co.KG, Ditzingen\\
            \textbf{Betreuer des Dualen Partners} & Manuel Geiger, M.Sc\\
    %       \textbf{Betreuer der Dualen Hochschule} & Name\\
        \end{tabular}
       
    \end{titlepage}

\pagenumbering{Roman}
\begingroup
\let\addcontentsline\relax
% \cleardoublepage
\thispagestyle{empty}   % keine Kopf-/Fußzeile auf dieser Seite
\pagestyle{empty}       % falls global fancy aktiv ist
\markboth{}{}           % Laufkopf-Markierungen leeren

\addchap*{Sperrvermerk}

 
 
Die vorliegende Projektarbeit beinhaltet interne und vertrauliche Informationen der Firma TRUMPF SE + Co. KG. Die Weitergabe des Inhalts, der Arbeit im Gesamten oder in Teilen, sowie Anfertigen von Kopien oder Abschriften, auch in digitaler Form, sind grundsätzlich untersagt. Ausnahmen bedürfen der schriftlichen Genehmigung durch Herrn Manuel Geiger, Betreuer dieser Projektarbeit bei TRUMPF SE + Co. KG Ditzingen.
\\[2em]
Dieser Sperrvermerk gilt zum \today.
 
\newpage
\cleardoublepage
\thispagestyle{empty}   % keine Kopf-/Fußzeile auf dieser Seite
\pagestyle{empty}       % falls global fancy aktiv ist
\markboth{}{}           % Laufkopf-Markierungen leeren

\addchap*{Eidesstattliche Erklärung}

 
Hiermit versichere ich, die vorliegende Projektarbeit selbstständig und nur unter Verwendung der von mir angegebenen Quellen und Hilfsmittel verfasst zu haben. Sowohl inhaltlich als auch wörtlich entnommene Inhalte wurden als solche kenntlich gemacht. Die Arbeit hat in dieser oder vergleichbarer Form noch keinem anderem Prüfungsgremium vorgelegen. \\
\\[2em]
Datum:  \hrulefill\enspace Unterschrift: \hrulefill

\newpage
\cleardoublepage
\thispagestyle{empty}   % keine Kopf-/Fußzeile auf dieser Seite
\pagestyle{empty}       % falls global fancy aktiv ist
\markboth{}{}           % Laufkopf-Markierungen leeren
% abstracts.tex — Deutsch & Englisch untereinander auf EINER Seite
\begin{samepage}
\section*{Kurzreferat}


\smallskip

\bigskip

\section*{Abstract}



\smallskip
\end{samepage}


\newpage

\endgroup

% \cleardoublepage
\thispagestyle{empty}   % keine Kopf-/Fußzeile auf dieser Seite
\pagestyle{empty}       % falls global fancy aktiv ist
\markboth{}{}           % Laufkopf-Markierungen leeren
% abstracts.tex — Deutsch & Englisch untereinander auf EINER Seite
\begin{samepage}
\section*{Kurzreferat}


\smallskip

\bigskip

\section*{Abstract}



\smallskip
\end{samepage}


\newpage



%%% Inhalts-, Abbildungs-, Tabellenverzeichnisse %%%
% werden einzeilig gesetzt, um Platz zu sparen 
\begin{spacing}{1}
\tableofcontents % Inhaltsverzeichnis ausgeben
\clearpage
\startAbkVerzeichnis

\begin{acronym}[DHBW] 
% Argument definiert die Breite der ersten Spalte anhand des längsten vorkommenden Eintrags
\acro{KI}{Künstliche Intelligenz}
\acro{2D}{Zweidimensional}
\acro{3D}{Dreidimensional}
\acro{mm}{Milimeter}
\acro{m}{Meter}
\acro{CNNs}{Convolutional Neural Networks}
\acro{ReLU}{Rectified Linear Unit}
\acro{QA}{Qualitätsschätzungsmodell}
\acro{VGG}{Visual Geometry Group}
\acro{Burr}{engl. Burr, Grat}
\acro{Roughness}{engl. Roughness, Rauheit}
\acro{ST}{engl. Steel, Baustahl}
\acro{SS}{engl. Stainless Steel, Edelstahl}
\end{acronym}

 % Abkürzungsverzeichnis einbinden

\clearpage
\thispagestyle{kapitelkopfzeile}
\listoffigures
\phantomsection
\addcontentsline{toc}{chapter}{\listfigurename} % Abb.verz. ins Inh.verz. aufnehmen

\clearpage
\listoftables
\phantomsection
\addcontentsline{toc}{chapter}{\listtablename} % Tab.verz. ins Inh.verz. aufnehmen
\end{spacing}

% Deutscher Name (optional)
\renewcommand{\lstlistlistingname}{Quellcodeverzeichnis}
\renewcommand{\lstlistingname}{Quellcode}

% Verzeichnis der Listings drucken
\lstlistoflistings

% (optional) ins Inhaltsverzeichnis aufnehmen
\addcontentsline{toc}{chapter}{\lstlistlistingname}


%%% Umstellung der Seiten-Nummerierung auf 1, 2, 3 ... %%%
\cleardoublepage
\pagenumbering{arabic}

%%% Ihr eigentlicher Inhalt %%%
% Empfehlung: strukturieren Sie Ihren Text in einzelnen Dateien 
% und binden Sie diese hier mit \input{includes/dateiname.tex} ein
% \chapter{Einführung}

% Dieses Kapitel gibt eine Einführung in die Thematik der Projektarbeit. Es werden die Zielsetzung und die geplante Vorgehensweise beschrieben. 

% \section{Zielsetzung}Das Ziel dieser Projektarbeit ist es, ein bestehendes Assistenzsystem auf Basis der \ac{KI} zu erweitern, das aktuell die Qualität beim Laserschneiden von Baustahlblechen vorhersagt und optimiert. Konkret soll die Leistungsfähigkeit dieses Modells auf Edelstahlbleche übertragen werden. Das aktuelle \ac{KI}-Modell weist in Bezug auf Edelstahl Defizite auf, da es bisher nur mit Datensätzen von Baustahl trainiert wurde und die spezifischen Eigenschaften von Edelstahl unzureichend berücksichtigt.Die Defizite sind insbesondere in der Vorhersagegenauigkeit der Schnittqualität bei Edelstahlblechen erkennbar.
% Um diese Lücke zu schließen, sollen neue, speziell auf Edelstahldaten generiert und in das bestehende Modell integriert werden. Diese Daten erfassen insbesondere typische Eigenschaften wie Schneidgratbildung und Oberflächenrauheit. Zusätzlich werden bestehende optische Messmethoden geprüft und gegebenenfalls angepasst, um ihre Eignung für Edelstahl sicherzustellen. Demnach können die neu generierten Daten vermessen und in die Datenbank für das \ac{KI}-Modell eingepflegt werden. Das Ziel ist ein robustes und zuverlässiges \ac{KI}-Modell, das die Qualität von Edelstahlschnitten ebenso präzise vorhersagen kann wie bereits für Baustahl.

% \section{Vorgehensweise}
% Zunächst wird ein systematischer Testplan erstellt, um wichtige Schneidparameter, insbesondere Laserleistung, Schnittgeschwindigkeit, Gasdruck und Fokuslage, für verschiedene Edelstahldicken zu untersuchen. Ziel dieses Testplans ist es, Parameterbereiche zu identifizieren, die zu sogenannten Schnittabrissen führen.Ein Schnittabriss entsteht, wenn durch ungünstige Schneidparameter das Werkstück nicht vollständig getrennt wird.
% Zur Begriffsabgrenzung vgl. die Definition „als das nicht vollständige Durchtrennen des Bleches“ \cite{Schindhelm2014}.

% Nachdem die kritischen Parameterbereiche, die zu schlechten Schneidergebnissen führen, identifiziert wurden, werden detaillierte Versuchspläne („Experimentalpläne“) zur Datengenerierung entwickelt. Diese umfassen systematische Schneidversuche an Edelstahlblechen mit Dicken bis zu 20 \ac{mm}. Dabei werden gezielt sowohl qualitativ hochwertige als auch minderwertige Schneidergebnisse erzeugt, um eine umfassende Datengrundlage für die Weiterentwicklung des KI-Modells bereitzustellen.

% Die generierten Schneidproben werden anschließend in einer Messzelle vermessen, um die resultierenden Schnittkanten in die KI-Datenbank aufzunehmen. Da die Messmethoden ursprünglich für Baustahl entwickelt wurden, müssen sie für Edelstahl angepasst werden. Dies betrifft insbesondere die Kalibrierung und Einrichtung des Handscanner-Setups, welches Bilder der Schnittkanten für die Qualitätsschätzung aufnimmt. Ebenso muss die Vektorberechnung des eingesetzten \ac{3D}-Punktwolkenscanners optimiert werden. Da Edelstahlschnittkanten typischerweise ausgeprägtere Schneidgrate aufweisen als die von Baustahlblechen, muss die Umrechnung der \ac{3D}-Punktwolke in einen \ac{2D}-Vektor entsprechend angepasst werden, um die tatsächlichen Merkmale der Schnittkanten präzise abzubilden.
% Neben der quantitativen Messung des Schneidgrats wird auch die Oberflächenrauheit qualitativ bewertet.
% Die aufbereiteten Messdaten fließen anschließend in die Erweiterung und das Training des bestehenden KI-Modells ein. Ziel ist, dass dieses Modell anschließend die Qualität der Laserschneidkanten bei Edelstahlblechen zuverlässig vorhersagen kann. Nach erfolgreicher Implementierung erfolgen Validierungstests sowie weitere gezielte Optimierungen, um die Vorhersagequalität kontinuierlich zu verbessern und sicherzustellen, dass das gewählte Parameterset bereits vor dem Schneidprozess zuverlässig bewertet werden kann.
\chapter{Grundlagen und Stand der Technik}
Für die vorliegende Arbeit sind Kenntnisse in den Bereichen Rapid Control Prototyping, modellbasierte Entwicklung mit Matlab und Simulink, Inertialsensorik sowie Lageschätzung mittels Sensorfusion erforderlich, weshalb in den folgenden Kapiteln die hierfür relevanten Grundlagen und der Stand der Technik dargestellt werden.


\section{Rapid Control Prototyping}
\label{sec:Rapid Control Prototyping}
"Tudor"

\section{MBD mit Matlab/Simulink}
\label{sec:MBD mit Matlab/Simulink}
"Tudor"

\newpage
\section{Quaternionen und Euler-Winkel}
\label{sec:Quaternionen und Euler-Winkel}

Für die Lageschätzung einer IMU ist eine Orientierungsdarstellung erforderlich, da die Lage der IMU aus den Messgrößen des Gyroskops, Beschleunigungssensors geschätzt und anschließend bewertet werden soll. Für die Umsetzung der Filteralgorithmen wird eine mathematische Beschreibung der Rotation benötigt.  
\\
Grundsätzlich kommen hierfür sowohl Quaternionen als auch Euler-Winkel in Frage. Quaternionen eignen sich besonders für die interne Berechnung, da sie eine kompakte und singularitätsfreie Darstellung von Rotationen ermöglichen. Zur Ausgabe und Visualisierung der Ergebnisse werden hingegen häufig Euler-Winkel verwendet, da sie die Orientierung anschaulich durch drei Winkel (Yaw, Pitch und Roll) beschreiben. Euler-Winkel besitzen jedoch Singularitäten und sind daher als interne Zustandsdarstellung nur eingeschränkt geeignet. 
\\

Euler-Winkel beschreiben die Orientierung durch drei aufeinanderfolgende Rotationen und werden üblicherweise als Yaw, Pitch und Roll angegeben. Abbildung \ref{fig:Eulerwinkel} zeigt diese Winkel als Rotationen um die körperfesten Achsen: Roll entspricht einer Rotation um die x-Achse, Pitch einer Rotation um die y-Achse und Yaw einer Rotation um die z-Achse. 


\begin{figure}[htbp]
  \centering
  \includegraphics[width=0.4\linewidth]{Eulerwinkel.png}
  \caption[Schematische Darstellung der Euler-Winkel]{Schematische Darstellung der Euler-Winkel\cite{ResearchGate}}
  \label{fig:Eulerwinkel}
\end{figure}


Die konkrete mathematische Beschreibung hängt von der gewählten Rotationsreihenfolge ab, da unterschiedliche Reihenfolgen zu unterschiedlichen Umrechnungsformeln zwischen Euler-Winkeln und Quaternionen führen. 

Ein zentrales Problem dieser Darstellung ist das Auftreten von Singularitäten. Für die Yaw-Pitch$-$Roll$-$Reihenfolge tritt der kritische Fall in der Nähe von Pitch {formel} auf, da zwei Rotationsachsen dabei effektiv zusammenfallen. In diesem Bereich sind Yaw und Roll nicht mehr unabhängig bestimmbar, sodass bereits kleine Änderungen der tatsächlichen Orientierung zu großen oder sprunghaften Änderungen einzelner Euler-Winkel führen können. Dieses Verhalten wird als Gimbal-Lock bezeichnet.\cite{KIT}

Quaternionen ... 
Formeln einbinden 

\newpage
\section{Lageschätzung mittels Sensorfusion}
\label{sec:Lageschätzung mittels Sensorfusion}








% Die in Tabelle~\ref{tab:symbole-rauheit-grat} zusammengefassten Formelzeichen werden in diesem Dokument verwendet.
% \begin{table}[htbp]
%   \centering
%   \ra{1.2}
%   \caption{Formelzeichen für Rauheit und Grat}
%   \label{tab:symbole-rauheit-grat}
%   \begin{tabular}{@{}lll@{}}
%     \toprule
%     Zeichen & Bedeutung & Einheit \\
%     \midrule
%     $Z(x)$      & gefiltertes Profil entlang der Auswertelänge $L$ & $\mu$m \\
%     $Z_i$       & diskreter Profilwert an Position $x_i$            & $\mu$m \\
%     $L$         & Auswertelänge                                      & mm \\
%     $n$         & Anzahl der Stützstellen in $L$                     & -- \\
%     $X_{s,j}$   & $j$-te Teilstrecke innerhalb von $L$               & mm \\
%     $R_a$       & arithmetischer Mittenrauwert                       & $\mu$m \\
%     $R_z$       & mittlere Rautiefe aus fünf Teilstrecken            & $\mu$m \\
%     $P^{\max}_{j}$, $P^{\min}_{j}$ & höchster bzw.\ tiefster Punkt in $X_{s,j}$ & $\mu$m \\
%     $h_b$       & Grathöhe an der unteren Schnittkante               & $\mu$m \\
%     \bottomrule
%   \end{tabular}
% \end{table}

% Die Berechnung der Kenngrößen folgt den nachfolgenden Rechenregeln. Der arithmetische Mittenrauwert \(R_a\) ist das Mittel der Beträge der Profilabweichung über die Auswertelänge \(L\) \parencite{MitutoyoQuickGuide}:
% \[
% R_a=\frac{1}{L}\int_{0}^{L}\lvert Z(x)\rvert\,\mathrm{d}x
% \]
% und in diskreter Form mit \(n\) Stützstellen \(Z_i\):
% \[
% R_a=\frac{1}{n}\sum_{i=1}^{n}\lvert Z_i\rvert.
% \]
% Die mittlere Rautiefe \(R_z\) wird über fünf Teilstrecken \(X_{s,1}\) bis \(X_{s,5}\) bestimmt \parencite{KeyenceISO4287}. In jeder Teilstrecke wird die Differenz zwischen höchstem und tiefstem Profilpunkt gebildet. \(R_z\) ist das arithmetische Mittel dieser fünf Differenzen:
% \[
% R_z=\frac{1}{5}\sum_{j=1}^{5}\bigl(P^{\max}_{j}-P^{\min}_{j}\bigr).
% \]

% Die Grathöhe \(h_b\) wird als maximale positive Auslenkung des Profils im Randbereich der unteren Schnittkante bestimmt. Grundlage ist dasselbe Profil \(Z(x)\) oder ein aus einer Punktwolke abgeleitetes Profil senkrecht zur Kante. Neben \(h_b\) können Breite und Form des Grates angegeben werden.
% Die Profilwerte \(Z_i\) stammen aus einer 3D-Punktwolke oder aus einer bildbasierten Profilerfassung. Aus diesen Werten werden \(R_a\) und \(R_z\) berechnet. Die Grathöhe \(h_b\) wird im Kantenbereich aus demselben Profil ermittelt. Einen Überblick zum verwendeten 3D-Messsystem gibt Abschnitt~\ref{sec:3d-messsystem-keyence}.

% Die Abbildung~\ref{fig:roughness-profile} zeigt das gefilterte Rauheitsprofil \(Z(x)\) über der Messstrecke \(X\).
% Die rote Linie kennzeichnet \(R_a\), \(Z_i\) sind die diskreten Profilwerte und \(X_{s,j}\) die Teilstrecken für \(R_z\). Die Grathöhe \(h_b\) wird an der unteren Schnittkante als größte positive Auslenkung im Kantenfenster bestimmt.



% \section{Convolutional Neural Networks (CNNs)}
% \label{sec:cnns}

% Das vorliegende soll ein Grundverständis zu \ac{CNNs} vermitteln, da dies die Grundlegende Architektur der neuronalen Netze ist und darauf der Cutting Assistant aufbaut (siehe Kapitel ~\ref{sec:cutting-assistent}).

% \ac{CNNs} sind neuronale Netze für Bilddaten. Sie bestehen aus wiederholten Blöcken aus Faltung (Convolution), nichtlinearer Aktivierung (z.\,B.\ \ac{ReLU}) und Pooling.
% Die wichtigsten Bausteine sind in den Unterkapiteln ~\ref{sec:cnns-conv-blocks}, ~\ref{sec:cnns-activation-batchnorm} und ~\ref{sec:cnns-pooling} näher beschrieben.

% Die Abbildung~\ref{fig:cnn-schema} zeigt den Aufbau eines \ac{CNNs} am Beispiel eines Roboterbildes. Zu Beginn laufen kleine Filter (typisch $3{\times}3$) über das Pixelgitter und reagieren auf lokale Muster, auch Kernel genannt. In den ersten Schichten entstehen somit Merkmalskarten für einfache Strukturen wie Kanten und Ecken, etwa an der Kontur des Kopfes oder entlang der Armsegmente des Roboters. Eine Aktivierungsfunktion unterdrückt schwache oder negative Antworten und erhöht den Kontrast zwischen relevanten und irrelevanten Bildbereichen. Pooling verdichtet die Merkmalskarten und macht die Darstellung unempfindlicher gegenüber kleinen Verschiebungen. Demnach ist die genaue Position der runden Augen oder Schrauben wird damit weniger wichtig als ihr Auftreten.

% Mit zunehmender Tiefe kombinieren weitere Faltungen diese einfachen Muster zu komplexeren Teilen wie Gesicht, Gelenken oder ganzen Körpersegmenten. Am Ende wird die verdichtete Repräsentation \emph{geflattet} und von vollverbundenen Schichten zu einer Ausgabe verdichtet, etwa zu Klassenwahrscheinlichkeiten (\enquote{Roboter ja/nein}) oder zu kontinuierlichen Werten. Das Netzwerk lernt dabei sämtliche Filtergewichte gemeinsam, sodass frühe und späte Schichten aufeinander abgestimmt sind und eine konsistente Merkmals­hierarchie vom Lokalen zum Globalen entsteht \parencite{FischerPochwyt_NeuronaleNetze_2017}.

% \begin{figure}[htbp]
%   \centering
%   \includegraphics[width=0.9\linewidth]{CNNs.png}
%   \caption[Schematischer Aufbau eines CNN]{Schematische Pipeline eines CNN am Beispiel eines Roboterbildes: Faltung extrahiert lokale Muster, Pooling verdichtet die Repräsentation, tiefere Stufen kombinieren Teile zu Objekten, die Entscheidung erfolgt in vollverbundenen Schichten \parencite{FischerPochwyt_NeuronaleNetze_2017}.}
%   \label{fig:cnn-schema}
% \end{figure}

% Nach demselben Prinzip arbeitet der in dieser Arbeit weiterentwickelte \enquote{Cutting Assistent} (siehe Kapitel~\ref{sec:cutting-assistent}). Frühe Filter reagieren auf Kanten und Texturwechsel in Schnittkantenbildern, Pooling sorgt für Robustheit gegen kleine Lageänderungen, und tiefere Schichten fassen wiederkehrende Muster wie Riefen, Rauheitsstrukturen oder Gratbildung zusammen. Die abschließenden Schichten liefern je nach Aufgabe eine Klassifikation oder Regressionswerte wie Rauheit oder Grathöhe.

% Die nachfolgenden Unterkapitel erläutern die zuvor genannten zentralen Bausteine eines \ac{CNNs} im Detail.

% \subsection{Faltungsblöcke}
% \label{sec:cnns-conv-blocks}
% Die 2D-Faltung berechnet an jeder Position \((i,j)\) einen gewichteten Mittelwert der lokalen Nachbarschaft. Die Gleichung~\eqref{eq:conv2d} definiert dies präzise. Der Kernel mit Kantenlänge \(k\) und Gewichten \(w_{u,v}\) wird zentriert über das Eingabebild \(x\) gelegt, die Offsets \((u,v)\) laufen über das Fenster. Das Ergebnis ist der Ausgabewert \(y_{i,j}\). Die in Tabelle ~\ref{tab:formelzeichen-min} aufgeführten Symbole legen die Notation fest. Durch das gemeinsame Nutzen der Kernelgewichte über alle \((i,j)\) sinkt die Parameterzahl und die Abbildung bleibt verschiebungsgleich. In praktischen Netzen ist die Faltung mehrkanalig, dies bedeutet dass mehrere Eingabekanäle gemeinsam gefaltet  werden und zu mehreren Ausgabekanälen kombiniert werden.
% % Kompakte Tabelle: nur die für \eqref{eq:conv2d} und \eqref{eq:gap} nötigen Formelzeichen
% \begin{table}[htbp]
%   \centering
%   \caption{Formelzeichen zu Gl.~\eqref{eq:conv2d} (Faltung) und Gl.~\eqref{eq:gap} (Global Average Pooling).}
%   \label{tab:formelzeichen-min}
%   \renewcommand{\arraystretch}{1.1}
%   \begin{tabular}{l p{0.62\linewidth} l}
%     \hline
%     \textbf{Symbol} & \textbf{Bedeutung} & \textbf{Einheit/Bereich} \\
%     \hline
%     $x_{i,j}$   & Eingabewert an Position $(i,j)$ & reell \\
%     $y_{i,j}$   & Ausgabewert der Faltung an Position $(i,j)$ & reell \\
%     $w_{u,v}$   & Gewicht des $k\times k$-Kerns an Offset $(u,v)$ & reell \\
%     $k$         & Kantenlänge des Faltungskerns & Pixel \\
%     $i,j$       & Räumliche Indizes (Zeile, Spalte) & — \\
%     $u,v$       & Kernelindizes & — \\
%     $H, W$      & Höhe und Breite der Merkmalskarte für GAP & Pixel \\
%     $z_c$       & GAP-Ausgabe für Kanal $c$ & reell \\
%     $c$         & Kanalindex & — \\
%     \hline
%   \end{tabular}
% \end{table}


% % 2D-Faltung (ein Kanal, Kernel k×k, zentriert)
% \begin{equation}
%   y_{i,j}
%   = \sum_{u=0}^{k-1}\sum_{v=0}^{k-1}
%     w_{u,v}\; x_{\,i+u-\lfloor k/2 \rfloor,\; j+v-\lfloor k/2 \rfloor}
%   \label{eq:conv2d}
% \end{equation}

% % Global Average Pooling (über H×W)
% \begin{equation}
%   z_c = \frac{1}{H\,W}\sum_{i=1}^{H}\sum_{j=1}^{W} y_{i,j,c}
%   \label{eq:gap}
% \end{equation}


% \subsection{Aktivierungsfunktion}
% \label{sec:cnns-activation-batchnorm}
% Die Faltung liefert lineare Antworten \(y_{i,j}\), die im nächsten Schritt nichtlinear transformiert werden. Eine Aktivierungsfunktion \(\phi(\cdot)\) wird punktweise angewandt und macht die Abbildung modellierfähig, etwa durch ReLU, die negative Werte auf null setzt.
% Die ReLU-Aktivierung ist eine üblich genutzte Aktiviewrungsfunktion.

% \subsection{Pooling und Downsampling (mit Global Average Pooling)}
% \label{sec:cnns-pooling}

% Pooling fasst lokale Nachbarschaften zusammen und reduziert die räumliche Auflösung. Dadurch sinkt der Rechenaufwand und die Darstellung wird unempfindlicher gegenüber kleinen Verschiebungen. Am Ende des Merkmalsextraktors wird häufig ein \emph{Global Average Pooling} eingesetzt. Die Gleichung ~\eqref{eq:gap} mittelt für jeden Kanal \(c\) alle Werte \(y_{i,j,c}\) über Höhe \(H\) und Breite \(W\) zu einem einzigen Skalar \(z_c\). Die so entstehende Vektordarstellung ist kompakt, senkt die Parameterzahl des Ausgabekopfes und eignet sich für Klassifikation und Regression. Andere Pooling-Varianten wie Max- oder lokales Average-Pooling können in früheren Stufen eingesetzt werden, ohne die in Gleichung ~\eqref{eq:gap} definierte globale Verdichtung am Ende zu ersetzen. Das Max-Pooling behält pro lokalem Fenster den größten Wert und hebt starke Aktivierungen hervor und das Min-Pooling behält den kleinsten Wert und betont schwache Antworten oder dunkle Strukturen. Beide Varianten reduzieren die Zahl der Werte und wirken als nichtlineare Verdichtung.

% \chapter{Stand der Technik}

% Dieser Stand der Technik ordnet die Arbeit in den aktuellen Forschungsstand ein. Inhaltlich wird der zuerweiternde KI-gestützte Cutting Assistant beschrieben. Außerdem wird das 3D-Kantenmesssystem erläutert, mit dem die Trainingsdaten erfasst werden.

% \section{KI-gestütztes Laserschneiden – Cutting Assistant}
% \label{sec:cutting-assistent}

% Der Cutting Assistant unterstützt die Qualitätssicherung und die Parametrierung beim Laserschneiden. Abbildung~\ref{fig:cutting_assistant} zeigt den Gesamtablauf von der Datenerfassung bis zur rückgekoppelten Optimierung der Prozessparameter. Zunächst erfasst einer am Schneidsystem montierter Handscanner die Schnittkante und übergibt die Bilddaten automatisiert an die Auswertekette. Ein vorgeschaltetes CNN (\enquote{Iron-hunter}, vgl. Abschnitt~\ref{sec:cnns}) prüft die Plausibilität der Aufnahme und verwirft Störbilder oder Fehlperspektiven. Beispiele für eine valide Kantenaufnahme und eine Fehlaufnahme sind in Abbildung~\ref{fig:ok-nok-examples} dargestellt. 

% \begin{figure}[!htbp]
%     \centering
%     \includegraphics[width=0.49\linewidth]{valid_edge_image.png}\hfill
%     \includegraphics[width=0.49\linewidth]{finger_false_image.png}
%     \caption{Beispiele zur Plausibilitätsprüfung durch \enquote{Iron-hunter}: links eine gültige Aufnahme der Schnittkante, rechts eine Fehlaufnahme mit verdeckender Hand, die verworfen wird.}
%     \label{fig:ok-nok-examples}
% \end{figure}

% Nur valide Daten gelangen zur Detektion, wo ein zweites Netz eine Qualitätsschätzung durchführt und Merkmale wie Rauheit und Grat schätzt. Die Kennzahlen fließen in ein Optimierungsmodell ein, wobei Gasdruck, Fokuslage und Schneidgeschwindigkeit anpasst werden. Wie diese Schneidparameter die Schneidqualität beeinflussen wird im oben Grundlagenkapitel ~\ref{sec:laserlichtschneiden} näher erläutert.

% Der Zyklus aus Aufnehmen, Bewerten und Anpassen wiederholt sich, bis ein Abbruchkriterium erfüllt ist, etwa wenn keine weitere Qualitätssteigerung zu erwarten ist oder die Spezifikation erreicht wurde. Das System ist für Baustahl validiert und wird im Rahmen dieser Arbeit auf Edelstahl erweitert, indem Datensätze ergänzt, Netzparameter feinjustiert und Schwellwerte an das Werkstoffverhalten angepasst werden.

% \begin{figure}[!htbp]
%     \centering
%     \includegraphics[width=0.95\linewidth]{cutting_assistant_pipeline.png}
%     \caption{Prozesskette des Cutting Assistant.}
%     \label{fig:cutting_assistant}
% \end{figure}



% \section{3D-Kantenmesssystem: Keyence LJ-X8060}
% \label{sec:3d-messsystem-keyence}

% In diesem Abschnitt wird die Funktionsweise des bereits eingesetzten dreidimensionalen Messsystems beschrieben, mit dem die Schnittkanten der Trainingsdaten für das KI-Modell erfasst wurden. Das System wurde zunächst für das Baustahlmodell verwendet und wird für die vorliegende Arbeit auf die Vermessung von Edelstahl angepasst. Die dafür notwendigen Optimierungen sind in Kapitel~\ref{chap:3d-punktwolken-optimierung} beschrieben.

% Abbildung~\ref{fig:lasertriangulation-schema} zeigt den Aufbau eines Sensors auf Basis der Lasertriangulation. Eine Sendeoptik projiziert eine Laserlinie auf die Oberfläche des Werkstücks. Versetzt zur Sendeeinheit erfasst eine Empfangsoptik das reflektierte Licht und bildet es auf einen zeilenförmigen Detektor ab. Die Position der Lichtlinie auf dem Detektor hängt von der Objektentfernung ab. Aus dieser Geometrie wird für jedes Zeilenprofil die Höhe des zugehörigen Oberflächenpunkts berechnet. Durch die Relativbewegung zwischen Sensor und Objekt entstehen aufeinanderfolgende Profile, die zu einer dreidimensionalen Punktewolke mit \((x,y,z)\) zusammengeführt werden.

% \begin{figure}[h!]
%     \centering
%     \includegraphics[width=0.4\linewidth]{keyence_triangulation.png}
%     \caption{Lasertriangulation zur Profilaufnahme. \parencite{SensorInstruments_LLASLT_Manual_2023}}
%     \label{fig:lasertriangulation-schema}
% \end{figure}

% Abbildung~\ref{fig:schnittkante-setup} fasst die Erfassung mit dem Keyence LJ-X8060 zusammen. Links ist der schematische Messaufbau mit Koordinatensystem gezeigt. Die Laserlinie tastet die Kante quer zur Scanrichtung ab und die Bewegung erfolgt entlang \(x\). Rechts ist eine registrierte dreidimensionale Punktewolke im Werkstückkoordinatensystem \((x,y,z)\) dargestellt. Für die Auswertung werden die Daten durch eine Koordinatenrotation so ausgerichtet, dass die Kante mit der \(x\)-Achse und die Blechebene mit der \(xy\)-Ebene zusammenfällt.

% \begin{figure}[h!]
%     \centering
%     \includegraphics[width=0.49\linewidth]{baustahlmessung_keyence.png}\hfill
%     \includegraphics[width=0.49\linewidth]{3D-Punktewolke.jpg}
%     \caption{Messaufbau und Beispielpunktwolke der Schnittkante für den Gratverlauf.}
%     \label{fig:schnittkante-setup}
% \end{figure}

% Aus der erfassten dreidimensionalen Punktewolke wird ein zweidimensionaler Vektor abgeleitet, der den Gratverlauf im \(yz\)-Schnitt beschreibt. Da die Schnittkante schräg erfasst wird, richtet eine Koordinatenrotation die Profile auf das Werkstückkoordinatensystem aus. Die Kante fällt danach mit der \(x\)-Achse zusammen und die Blechebene mit der \(xy\)-Ebene. Die Rauheit der Schnittfläche wird als zweidimensionaler Vektor in der \(xy\)-Ebene beschrieben. Die Auswertung erfolgt entlang \(x\) und wird über \(y\) gemittelt. Eine ausführliche Beschreibung der Berechnung von Grat und Rauheit ist in Abschnitt~\ref{sec:grat-rauheit} zu finden.

\chapter{Datengenerierung von Edelstahldatensätzen}

In dem folgendem Kapitel wird die Versuchsplanung zur Identifikation von Schnittabrissgrenzen sowie die Messmethodik und Datenerfassung in der Messzelle für Edelstahlbleche beschrieben. Ziel ist es, den prozesssicheren Arbeitsbereich beim Laserschneiden von Edelstahl systematisch abzugrenzen und eine belastbare Datenbasis für die Erweiterung des KI-gestützten Laserschneidassistenten aufzubauen. Anschließend wird die Anpassung des Handscanner-Setups erläutert, um die Bildqualität für die bildgestützte Qualitätsschätzung zu optimieren.

\section{Bestimmung der Schnittabrissgrenzen und Erstellung des Datensatzes}
Ziel dieses Kapitels ist die systematische Abgrenzung des prozesssicheren Arbeitsbereichs beim Laserschneiden von Edelstahlblechen sowie die präzise Identifikation der Parameterbereiche, in denen Schnittabrisse auftreten. Unter einem Schnittabriss wird im Folgenden ein Zustand verstanden, in dem das Werkstück infolge ungeeigneter Parameterkombinationen nicht vollständig getrennt wird, weil der Schnittspalt lokal verschweißt oder die Schmelzaustragung unzureichend ist. Die auf diese Weise gewonnenen Grenzwerte bilden die Grundlage für belastbare Parametrierungsempfehlungen und fließen zugleich in die Ausarbeitung strukturierter Versuchspläne zur Datengenerierung ein.

Die Experimente sind als zweidimensionale Rasterstudien ausgelegt, bei denen jeweils zwei der drei wesentlichen Prozessparameter, in dem Fall der Arbeit ist es die Fokuslage, die Schnittgeschwindigkeit und dem Gasdruck, variiert werden, während der dritte Parameter konstant gehalten wird. Die Laserleistung bleibt in diesen Studien konstant. Für jede untersuchte Parameterpaarung wird ein $3\times 3$-Feld gefertigt, in dem eine Größe entlang der horizontalen und die andere entlang der vertikalen Richtung stufenweise verändert wird. Formal seien die Stufen der beiden variierten Parameter $x\in\{x_1,x_2,x_3\}$ und $y\in\{y_1,y_2,y_3\}$. Der jeweils dritte Parameter $z$ ist auf einem Referenzniveau $z_0$ fixiert. Die Studie wird sequenziell für alle drei Kombinationen wiederholt, sodass das Prozessfenster in den relevanten Teilräumen konsistent erfasst wird. Die Wahl der Stufen erfolgt material- und dickenspezifisch, aus praxisüblichen Startwerten und internen Erfahrungswerten wird ein plausibler Arbeitsbereich abgeleitet.

Für jede Parameterkombination wird ein Viereck geschnitten. Ein Schnitt gilt als erfolgreich, wenn der Trennschnitt vollständig ist, weder Durchhang noch Wiederaufschmelzen im Schnittspalt beobachtet wird und der Schmelzaustrag kontinuierlich erfolgt. Die Beurteilung erfolgt unmittelbar an der Maschine sowie nachgelagert in der Messzelle durch optische Inspektion und Dokumentation der Schnittkante. Sobald ein erster Grenzbereich identifiziert ist, wird das umliegende Parametergebiet gezielt erkundet, bis ein stabiler Übergang zwischen den Zuständen „Schnitt möglich“ und „Schnittabriss“ reproduzierbar nachgewiesen ist. Die so gewonnenen Grenzpunkte werden im jeweiligen Parameterraum verortet und bilden eine aus Messdaten abgeleitete Näherung des Prozessfensters je Blechdicke.

Auf Basis dieser Grenzanalysen erfolgt die Datengenerierung in Form strukturierter Experimentalpläne. Hierfür wurden insgesamt 17 Edelstahlblechtafeln (1\,\ac{m} $\times$ 2\,\ac{m})  in den Dicken 5\,mm, 8\,mm, 10\,mm, 15\,mm und 20\,mm eingesetzt. Auf jeder Tafel wurden 128 quadratische Proben (100\,mm $\times$ 100\,mm) geschnitten, sodass ein Gesamtdatenumfang von 2\,176 Bauteilen entstand. Die Schneidparameter wurden pro Bauteil innerhalb der vorab definierten, dickenspezifischen Grenzen variiert und nach einem vorgegebenen Schema zufällig ausgewählt. Dieses Design stellt sicher, dass das Datenset sowohl hochwertige als auch ausdrücklich minderwertige Schneidergebnisse enthält, einschließlich Fehlschnitten und Parameterkombinationen nahe der Schnittabrissgrenze. Solche Negativbeispiele sind für die spätere Modellierung essentiell, um die Trennschärfe zwischen »prozesssicher« und »instabil« zu erhöhen und Fehlklassifikationen zu vermeiden.

Fehlschnitte wurden vollständig protokolliert. Auch wenn betroffene Proben in Einzelfällen nicht aus der Großtafel entnommen werden konnten, gingen diese Versuche mit eindeutiger Kennzeichnung in die Datenbank ein; damit ist bekannt, dass die jeweilige Parameterkombination für die gegebene Blechdicke kein akzeptables Schneidergebnis liefert. Sämtliche entnehmbaren Bauteile werden in der Messzelle vermessen, identifiziert und mit ihren Soll-/Ist-Parametern verknüpft. Die Datenmenge und der Versuchsplan wurden auf Basis der Erfahrungen mit dem bereits für Baustahl trainierten Modell gewählt, um eine ausreichende Abdeckung des Parameterraums und eine robuste Generalisierungsfähigkeit für Edelstahl sicherzustellen.

In Summe ermöglicht die kombinierte Vorgehensweise aus Grenzbestimmung und gezielter Datengenerierung sowohl die belastbare Identifikation der Schnittabrissgrenzen als auch den Aufbau einer ausgewogenen Datenbasis. Diese bildet die Voraussetzung für die Erweiterung und Validierung des KI-Modells, das künftig die Qualität von Edelstahlschnitten prädiktiv bewerten und prozesssichere Parameterbereiche verlässlich empfehlen soll.

\section{Messmethodik und Datenerfassung in der Messzelle}


Die Messzelle dient der reproduzierbaren Erfassung aller Messdaten zu den im Rahmen der Experimentalpläne geschnittenen Edelstahlbauteilen. Sie ist als sequenzieller Messprozess ausgelegt, in dem ein mehrachsiger KUKA-Industrieroboter die Proben zwischen den Stationen handhabt. Die Bauteile werden an der Startposition gestapelt bereitgestellt, vom Roboter mittels Vakuumgreifer aufgenommen und der ersten Station zugeführt. Dort erfolgt die automatisierte Probenidentifikation über einen aufgebrachten QR-Code (siehe ID-Lesegerät in Abbildung ~\ref{fig:handscanner_barcodelesegerät_keyence} und Edelstahlprobe mit QR-Code in Abbildung ~\ref{fig:probenID}). Die ermittelte Proben-ID wird mit den Metadaten aus den Experimentalplänen (z.\,B. Blechdicke, Soll-Parameter) verknüpft und dient in der Folge als Schlüssel für die Mess- und Auswertedaten.

Im Anschluss werden an einer Station hochaufgelöste Aufnahmen der ersten Schnittkante erfasst. Hierzu kommt ein Handscanner (siehe Abbildung ~\ref{fig:handscanner_barcodelesegerät_keyence})zum Einsatz, dessen Aufnahmeparameter, wie z.B. Arbeitsabstand, Belichtung und  Auflösung, konstant gehalten werden. Diese Bilddaten bilden die Grundlage für die bildgestützte Qualitätsschätzung des KI-Systems. Ergänzend dazu wird die gleiche Schnittkante mit einem Keyence-3D-Messsystem (siehe Abbildung ~\ref{fig:handscanner_barcodelesegerät_keyence}) dreidimensional vermessen, sodass eine 3D-Punktwolke des Kantenverlaufs entsteht. Aus dieser Punktwolke werden definierte Profilverläufe abgeleitet und geometrische Kenngrößen berechnet, die der Erfassung von Gratbildung und der Oberflächenrauheit dienen. Die Berechnung des Grates und der Rauheit ist im obigen Grundlagenkapitel ~\ref{sec:grat-rauheit} näher erläutert. Die genaue funktionsweise des 3D-Messsystems ist im folgenden Kapitel ~\ref{sec:3d-messsystem-keyence} ausgiebig erläutert. Die so gewonnenen Ist-Kenngrößen fungieren als Referenz für den späteren Abgleich mit der Bildqualitätsschätzung. In der Abbildung ~\ref{fig:handscanner_barcodelesegerät_keyence} sind die im obigen Abschnitt beschriebenen Komponenten der Messzelle während einer Messung.

In der folgenden Abbildung ~\ref{fig:handscanner_barcodelesegerät_keyence} sind die im obigen Abschnitt beschriebenen Komponenten der Messzelle während einer Messung dargestellt.
\begin{figure}[htbp]
    \centering
    \includegraphics[width=0.6\linewidth]{handscanner_keyence.jpg}
    \caption{Handscanner (1), ID-Lesegerät (2), Keyence 3D-Messsystem (3) in der Messzelle}
    \label{fig:handscanner_barcodelesegerät_keyence}    
\end{figure}

\newpage

Zur vollständigen Dokumentation werden die Schnittkanten zudem mit einer Industriekamera und einem stationären Smartphone-Setup unter verschiedenen Beleuchtungsbedingungen aufgenommen. Die Kombination aus unterschiedlichen Kameras und Beleuchtungen erhöht die Robustheit der visuellen Beurteilbarkeit und unterstützt die spätere manuelle Nachvollziehbarkeit von Auffälligkeiten. Die erfassten Messdaten des Werkstücks, sowie die daraus abgeleiteten Kenngrößen der Proben-ID zugeordnet und in die zentrale Datenbank überführt.

Die Abbildung~\ref{fig:smartphone_industriekamera} zeigt die zuvor beschriebenen Komponenten der Messzelle. Oben im Bild ist das Smartphone (4) mit LED-Ringlicht zu sehen, welches die Schnittkante aus einem 90° Winkel aufnimmt. Unten im Bild ist die Industriekamera (5) mit Auflichtbeleuchtung dargestellt, welche die Schnittkante ebenfalls aus einem 90° Winkel erfasst. Beide Kameras sind fest in der Messzelle montiert und werden automatisch durch den Roboter angesteuert.
\begin{figure}[htbp]
    \centering
    \includegraphics[width=0.6\linewidth]{smartphone_industriekamera.jpg}
    \caption{Smartphone (4) und Industriekamera (5) in der Messzelle}
    \label{fig:smartphone_industriekamera}
\end{figure}

Nach der Datenerfassung werden aus der 3D-Messung die tatsächlichen Kenngrößen der Schnittkante berechnet und den Ergebnissen der bildbasierten Qualitätsschätzung gegenübergestellt. Dieser Abgleich ermöglicht die Beurteilung der Übereinstimmung zwischen qualitativer, bildgestützter Bewertung und quantitativer Geometriemessung. Die beschriebenen Messschritte werden für alle vier Schnittkanten jedes Bauteils identisch wiederholt. Abschließend legt der Roboter die vollständig vermessenen Proben an der Endstation geordnet ab.

Anbei ist ein besipielhafter Vergleich der geschätzten Kenngrößen einer Schnittkante im Vergleich zu den gemessenen Kenngrößen in den Abbildungen  ~\ref{fig:burr_true_pred} und ~\ref{fig:roughness_true_pred} dargestellt. Es sind jeweils zwei Diagramme zu sehen, einer für den Gratwert und einer für die Rauheit. Auf den y-Achsen sind die geschätzten Kenngrößen und auf der x-Achse die gemessenen Kenngrößen aufgetragen. Idealerweise liegen alle Punkte auf der Diagonalen, was eine perfekte Übereinstimmung zwischen Schätzung und Messung bedeuten würde. In disem Beispiel Überschätzt das KI-Modell den Rauheitswert leicht, whärend der Gratwert überwiegend "gut" geschätzt wird. Mit einer leichten Abweichung ist stets zu rechnen, da die bildbasierte Schätzung eine Näherung darstellt und nicht alle Details der 3D-Messung erfassen kann.

\begin{figure}[h!]
    \centering
    \begin{minipage}{0.48\textwidth}
        \centering
        \includegraphics[width=\linewidth]{qualitatsschatzung_burr.png}
        \caption{Vergleich zwischen vorhergesagten und tatsächlichen Werten für den Grat (Burr).}
        \label{fig:burr_true_pred}
    \end{minipage}
    \hfill
    \begin{minipage}{0.48\textwidth}
        \centering
        \includegraphics[width=\linewidth]{qualitatsschatzung_roughness.png}
        \caption{Vergleich zwischen vorhergesagten und tatsächlichen Werten für die Rauheit (Roughness).}
        \label{fig:roughness_true_pred}
    \end{minipage}
\end{figure}



\newpage

\subsection{Anpassung der Handscanner Einstellungen für Edelstahl}
\label{chap:handscanner-setup}

Für die bildgestützte Qualitätsschätzung werden die mit dem Handscanner aufgenommenen Schnittkantenbilder als zentrale Eingangsgröße verwendet. Die bisher im Einsatz befindlichen Aufnahmeparameter waren für Baustahl optimiert. Baustahl weist im Vergleich zu Edelstahl eine geringere Oberflächenreflexion und eine tendenziell matte Erscheinung auf. Werden diese Einstellungen unverändert auf Edelstahl angewandt, führt die höhere Reflexion zu Bildartefakten und zu einer unzureichenden Abbildung der relevanten Mikrostruktur \parencite{ZahnerStainlessReflect}. In der Folge würden Grate (\emph{engl. Burr}) unterrepräsentiert und die Rauheit (\emph{engl. Roughness}) potenziell verfälscht erscheinen. Da die Klassifikation der Bildqualität und die darauf basierende Schätzung von \emph{Burr} und \emph{Roughness} unmittelbar in die Parametrierung des Laserschneidprozesses zurückwirken, ist eine werkstoffabhängige Anpassung des Handscanner-Setups zwingend erforderlich.

Das Aufnahmeprotokoll sieht pro Schnittkante drei Bilder vor: (i) ein bewusst dunkler belichtetes Bild, das primär der Beurteilung der Schnittflächenrauheit dient, sowie (ii) zwei überbelichtete Bilder, die gemeinsam mit dem ersten zu einem HDR-Komposit zusammengeführt werden, um die Kontur und Ausprägung des Grats sicher zu erfassen (siehe besipielhafte Aufnahme in Abbildung ~\ref{fig:vier_bilder_simple}). 

\begin{figure}[htbp]
  \centering
  \includegraphics[width=.24\textwidth]{A1221E0233-H5uCOXjtm6-3-100-0011_20250828083003-1_HACO_OFFOFFON_625_12.0_3.png}\hfill
  \includegraphics[width=.24\textwidth]{A1221E0233-H5uCOXjtm6-3-100-0011_20250828083003-1_HACO_OFFOFFON_1400_12.0_2.png}\hfill
  \includegraphics[width=.24\textwidth]{A1221E0233-H5uCOXjtm6-3-100-0011_20250828083003-1_HACO_ONONOFF_270_12.0_1.png}\hfill
  \includegraphics[width=.24\textwidth]{A1221E0233-H5uCOXjtm6-3-100-0011_20250828083003-1_HACO_burr_HDR.png}
  \caption{Dunkle Belichtung für die Rauheitsschätzung (i, erstes Bild links), zwei helle Belichtungen für die Gratschätzung (ii, mittlere Bilder) und das daraus generierte HDR-Bild (rechts).}
  \label{fig:vier_bilder_simple}
\end{figure}


Die Kalibrierung der Belichtung erfolgt schrittweise. Zunächst wird die Belichtungszeit für das Rauheitsbild so eingestellt, dass die Textur der Schnittfläche ohne Sättigung und mit klarer Detailzeichnung sichtbar ist. Diese Entscheidung erfolgt in dieser Phase bewusst subjektiv, jedoch anhand vorab definierter visueller Kriterien, wie z.B. ausreichender Tonwertumfang und erkennbarer Strukturkontrast. Im Anschluss werden die Belichtungsparameter der beiden HDR-Bilder iterativ variiert, bis der Grat entlang der Schnittkante über den gesamten Bildbereich eindeutig detektierbar ist, ohne dass umliegende Bereiche vollständig verloren gehen. Da die HDR-Komposition durch die Eingangsbilder beeinflusst wird, erfolgt die Abstimmung der HDR-Belichtungen stets nach der Festlegung des Rauheitsbildes. 

Zur Sicherstellung der Kompatibilität mit dem bestehenden KI-Modell wird die Anpassung an Referenzaufnahmen aus der bereits validierten Baustahlkonfiguration ausgerichtet. Praktisch bedeutet dies, dass eine Baustahlschnittkante mit den etablierten Baustahleinstellungen aufgenommen wird und die Edelstahlaufnahmen so justiert werden, dass die resultierenden Bildcharakteristika in qualitativer Hinsicht vergleichbar sind. Auf diese Weise wird gewährleistet, dass die Edelstahlbilder in das bestehende Modell eingebunden und mit den vorhandenen Trainings- und Bewertungsroutinen verarbeitet werden können.

Die Abbildungen~\ref{fig:burr_baustahl},~\ref{fig:burr_edelstahl}, ~\ref{fig:roughness_baustahl} und ~\ref{fig:roughness_edelstahl} zeigen exemplarisch die finalen Handscanner Bilder, welche für die Qualitätsschätzung genutzt werden, für Edelstahl im Vergleich zu den bisherigen Bildern für die Qualitätsschätzung von Baustahl. In den Abbildungen ist jeweils die gleiche Schnittkante eines Blechteils aus den geschnittenen Experimentalplänen dargestellt mit einer Dicke von 15 mm, einmal ein Baustahlblech und einmal ein Edelstahlblech.

\begin{figure}[htbp]
  \centering
  \begin{minipage}{0.48\linewidth}
    \centering
    \includegraphics[width=\linewidth]{burr_baustahl.png}
    \caption{Handcanner Bild für die Gratschätzung (Baustahl)}
    \label{fig:burr_baustahl}
  \end{minipage}\hfill
  \begin{minipage}{0.48\linewidth}
    \centering
    \includegraphics[width=\linewidth]{burr_edelstahl.png}
    \caption{Handcanner Bild für die Gratschätzung (Edelstahl)}
    \label{fig:burr_edelstahl}
  \end{minipage}
\end{figure}

\begin{figure}[htbp]
  \centering
  \begin{minipage}{0.48\linewidth}
    \centering
    \includegraphics[width=\linewidth]{roughness_baustahl.png}
    \caption{Handscnanner Bild für die Rauheitsschätzung (Baustahl)}
    \label{fig:roughness_baustahl}
  \end{minipage}\hfill
  \begin{minipage}{0.48\linewidth}
    \centering
    \includegraphics[width=\linewidth]{roughness_edelstahl.png}
    \caption{Handscanner Bild für die Rauheitsschätzung (Edelstahl)}
    \label{fig:roughness_edelstahl}
  \end{minipage}
\end{figure}

Zur konsistenten Anwendung der angepassten Handscanner-Parameter wird das Messzellen-Skript so erweitert, dass das passende Setup automatisiert auf Basis der Bauteilbezeichnung gewählt wird. Die Benennung folgt dem Schema
\texttt{Maschinenname-\allowbreak Experimentalplanname-\allowbreak Materia
lnummer-\allowbreak Bauteildicke-\allowbreak Bauteilnummer},
z.\,B.\ \texttt{A02280E0005-\allowbreak AiMuWrCjd0-\allowbreak 3-\allowbreak 050-\allowbreak 0176}.
Die folgende Abbildung ~\ref{fig:probenID} zeigt eine Beispiel-Proben-ID eines Blechstücks aus den Experimentalplänen mit den einzelnen Segmenten.

\begin{figure}[htbp]
    \centering
    \includegraphics[width=0.7\linewidth]{Werkstück_Proben-ID.jpg}
    \caption{Beispiel-Proben-ID eines Blechstücks aus den Experimentalplänen}
    \label{fig:probenID}
\end{figure}

Das Skript parst die Zeichenkette, prüft die Zahl nach dem zweiten Bindestrich und lädt abhängig davon die vordefinierten Handscanner-Einstellungen für den jeweiligen Werkstoff. Auf diese Weise wird sichergestellt, dass die für Edelstahl kalibrierten Belichtungen und Aufnahmeparameter reproduzierbar zur Anwendung kommen und die so erzeugten Bilder ohne systematische Verzerrungen in die Qualitätsmodellierung eingehen. Dies ist im folgendem C-Sharp Quellcode ~\ref{lst:messzellen-routing} dargestellt und im Messzellenskript inplementiert.

% \chapter{Optimierung der Vektorberechnung aus der 3D-Punktwolken}
% \label{chap:3d-punktwolken-optimierung}

% Die mit dem Keyence-System aufgenommene 3D-Punktwolke der Schnittkante bildet die Grundlage für die geometrische Qualitätsauswertung. In der bestehenden Auswertepipeline wird die Punktwolke zunächst segmentiert und in ein lokales Kantenkoordinatensystem überführt. Anschließend erfolgt eine Projektion aus der dreidimensionalen Repräsentation in einen zweidimensionalen Profilverlauf, sodass ein 2D-Vektor entsteht, der den Verlauf des Schneidgrats entlang der Schnittkante beschreibt. Diese Vorgehensweise wurde ursprünglich für Baustahl entwickelt und auf dessen charakteristisch eher wellige, kontinuierliche Gratmorphologie abgestimmt.

% Die nachfolgend beschriebenen Anpassungen betreffen ausschließlich den \textbf{Grat-/Burr}-Verlauf unterteilt in der Ausreißererkennung von Messdaten und der darauffolgenden Interpolation. Analyse ob die Rauheitsanalyse des Baustahl-Modells auch auf Edelstahl erfolgen kann wird in Kapitel ~\ref{sec:roughness-analysis} beschrieben.

% \section{Gratverlaufsanalyse bei Edelstahl}

% Bei Edelstahl zeigt sich eine abweichende, ausgeprägt zackige Gratstruktur mit höheren lokalen Gradienten und diskontinuierlichen Profilabschnitten. Die bislang implementierte Outlier-Korrektur ist konzipiert zur Eliminierung sporadischer Messfehler bei Baustahl. Demnach stuft diese den Gratverlauf von Edelstahl fälschlich als Ausreißer ein und glättet sie übermäßig. Dadurch werden relevante Merkmale des Edelstahlgrats unterdrückt und der resultierende Vektorverlauf in Richtung eines künstlich „glatten“ Profils verzerrt. Den unterschiedlichen Gratverlauf ist in den obigen Handscanneraufnahmen von Baustahl und Edelstahl ersichtlich (siehe Abbildungen ~\ref{fig:burr_baustahl} und ~\ref{fig:burr_edelstahl}).

% Erschwerend kommt hinzu, dass im Messprozess partiell überbeschattete Bereiche auftreten können, die vom Sensor nicht erfasst werden. In der bisherigen Pipeline werden solche Lücken durch Interpolation geschlossen, deren Parameter auf die kontinuierlichen Profile von Baustahl zugeschnitten sind. Für den zackigen Edelstahlgrat führt dies zu einer zu starken Annäherung an glatte Zwischenverläufe und damit zu einem Verlust an formcharakteristischer Information.

% Zur materialspezifischen Anpassung werden daher zwei Kernmodule überarbeitet. Die Ausreißererkennung \cref{sec:outlier-correction} mit nachgelagerter Korrektur und die Interpolation \cref{sec:Interpolation} fehlender Stützstellen. In der Ausreißererkennung werden die Schwellwerte und die zugrunde liegenden Sensitivitätsmaße an die höhere lokale Krümmung und den gesteigerten Kantenkontrast des Edelstahlgrats angepasst. Ziel ist eine Differenzierung zwischen echten Messfehlern und materialtypischen Hochfrequenzanteilen. Entsprechend werden Glättungsschritte zurückgenommen, sodass signifikante Gratflanken erhalten bleiben. 

% Für die Interpolation wird ein konservativer Ansatz gewählt, der Lückenschlüsse bevorzugt entlang lokal konsistenter Nachbarschaften vornimmt und globale, stark glättende Approximationen vermeidet. Damit wird erreicht, dass der rekonstruierte 2D-Vektor fehlende Messpunkte plausibel ergänzt, ohne die charakteristische Zackigkeit des Edelstahlgrats zu nivellieren.

% \subsection{Optimierung der Ausreißererkennung}
% \label{sec:outlier-correction}

% Ziel der Ausreißerbehandlung im \textbf{Burr}-Verlauf ist die zuverlässige Trennung zwischen echten Messfehlern und materialtypischen Hochfrequenzanteilen des Edelstahlgrats. Die Verarbeitung ist zweistufig aufgebaut. Zuerst erfolgt ein Vergleich des gemessenen Profils mit einem lokal geglätteten Referenzsignal zur Detektion potenzieller Ausreißerpunkte. Anschließend werden entstandene Lücken kontrolliert rekonstruiert. Der zugehörige Pseudocode ist in \cref{alg:simple-burr} dargestellt.

% In \texttt{outlier\_correction\_burr} dient ein gleitender Mittelwert auf \texttt{z\_vec} als Referenzsignal. Die punktweise Abweichung \(\Delta=\lvert z-\overline{z}_{\text{MA}}\rvert\) wird mit dem Schwellwert \texttt{threshold} verglichen. Alle Punkte oberhalb dieses Schwellwertes bilden die Ausreißermaske. Überschreitet der Anteil markierter Punkte den Grenzwert \texttt{max\_nan\_values\_perc}, wird der Lauf verworfen und \texttt{None} zurückgegeben. Dieses Abbruchkriterium wirkt als Qualitätswächter bei stark gestörten Profilen. Liegt der Anteil darunter, werden die Ausreißer in \texttt{z\_vec} zu \texttt{NaN} gesetzt und mit \texttt{smooth\_nan\_values} rekonstruiert. Die Rekonstruktion schließt kurze Lücken entlang der lokalen Nachbarschaft und erhält charakteristische Gratflanken, sodass der resultierende Vektor die materialspezifische Zackigkeit beibehält.

% Die Parametrisierung orientiert sich an der spektralen Struktur des Profils und an der Messrauschamplitude. Das Fenster \texttt{window} bestimmt die lokale Glättung des Referenzsignals. Ein kleineres Fenster reagiert sensibler auf kurzwellige Änderungen, während ein größeres Fenster stärker glättet. Der Schwellwert \texttt{threshold} definiert die Minimalhöhe einer Abweichung, ab der ein Punkt als Ausreißer gilt. Für Edelstahl wird ein höherer Schwellwert gewählt, um die ausgeprägten kurzwelligen Strukturen nicht fälschlich zu maskieren. Der Grenzwert \texttt{max\_nan\_values\_perc} begrenzt die zulässige Ausreißerquote und verhindert Rekonstruktionen auf unzureichender Datenbasis. Die konkreten Werte für Edelstahl und die Referenzkonfiguration für Baustahl sind in \cref{lst:params-outlier-stainless} und \cref{lst:params-outlier-mild} angegeben.

% Aus methodischer Sicht besitzt die Vorgehensweise eine lineare Laufzeit in der Profil­länge und ist damit für große Punktzahlen geeignet. Die Nutzung eines zentrierten gleitenden Mittelwerts sorgt für eine symmetrische Referenzbildung. Randbereiche werden durch die Faltung im Same-Modus in der Länge konsistent gehalten. Längere Messausfälle werden bewusst nicht vollständig interpoliert, da dies die Formtreue des Burr-Verlaufs beeinträchtigen würde. In solchen Fällen greift das Abbruchkriterium und markiert die entsprechende Profilzeile für eine erneute Datenerfassung oder für eine angepasste Parametrisierung.

% \begin{algorithm}[H]
% \DontPrintSemicolon
% \caption{Einfache Ausreißer-Korrektur im Burr-Profil \,(vgl. Parameter in \cref{lst:params-outlier-stainless,lst:params-outlier-mild})}
% \label{alg:simple-burr}
% \KwIn{Positionen $x$, Höhen $z$, Schwelle $\tau$, Fenster $w$, max.~Anteil $\alpha$}
% \KwOut{$(x,\ z_{\text{clean}})$ oder \texttt{None}}

% % 1) Referenzkurve
% ref $\leftarrow$ MOVING\_AVERAGE($z$, $w$, centered)\;   % gleitender Mittelwert

% % 2) Abweichung und Maske
% diff $\leftarrow$ ABS($z$ $-$ ref)\;
% mask $\leftarrow$ (diff $>$ $\tau$)\;
% anteil $\leftarrow$ COUNT(mask) / LENGTH($z$)\;

% % 3) Abbruch bei zu vielen Ausreißern
% \If{anteil $>$ $\alpha$}{\Return (\texttt{None}, \texttt{None})}

% % 4) Ausreißer entfernen und Lücken glätten
% set $z$[mask] $\leftarrow$ NaN\;
% $z_{\text{clean}}$ $\leftarrow$ SMOOTH\_NAN\_VALUES($x$, $z$)\; % kurze Lücken füllen

% \Return ($x$, $z_{\text{clean}}$)
% \end{algorithm}

% In der Edelstahl-Pipeline wurden der Schwellwert der Burr-Ausreißererkennung angehoben und das Fenster leicht verkürzt. Ziel ist die Erhaltung materialspezifischer Hochfrequenzanteile bei zugleich robuster Erkennung echter Messfehler. Die verwendeten Parameter sind in \cref{lst:params-outlier-stainless} dokumentiert. Zum Vergleich ist die bisherige Baustahl-Konfiguration in \cref{lst:params-outlier-mild} angegeben. Für die Rauheitsanalyse wurden keine Parameter geändert.

% \begin{lstlisting}[caption={Pipeline-Parameter Outlier Correction (Edelstahl, Burr-Verlauf) \,(zu \cref{alg:simple-burr})}, label={lst:params-outlier-stainless}]
% # Parameter burr outlier correction (Edelstahl)
% burr_outlier_threshold : 0.06  # Threshold for Moving Average Difference Filter [mm]
% burr_outlier_window    : 9     # Window for Moving Average Difference Filter [samples]
% \end{lstlisting}

% \begin{lstlisting}[caption={Pipeline-Parameter Outlier Correction (Baustahl, Burr-Verlauf) \,(zu \cref{alg:simple-burr})}, label={lst:params-outlier-mild}]
% # Parameter burr outlier correction (Baustahl)
% burr_outlier_threshold : 0.03  # Threshold for Moving Average Difference Filter [mm]
% burr_outlier_window    : 10    # Window for Moving Average Difference Filter [samples]
% \end{lstlisting}

% \subsection{Optimierung der Interpolation}
% \label{sec:Interpolation}

% Die Interpolation rekonstruiert fehlende Messwerte (\texttt{NaN}), die durch Ausreißerkennzeichnung oder unvollständige Erfassung entstehen. Ziel ist die Wiederherstellung eines plausiblen Gratverlaufs, daher werden nur kurze, lokal begrenzte Lücken gefüllt und größere Ausfälle bleiben markiert. Durch die angepasste Ausreißererkennung werden weniger echte Messpunkte fälschlich als Ausreißer markiert, sodass deutlich weniger Lücken entstehen und nicht mehr unnötig interpoliert wird. Am Interpolationsskript wurden keine Änderungen vorgenommen. Alle Konstanten bleiben unverändert, insbesondere die maximale Lückenlänge sowie Glättungs- und Fensterparameter. Der zugehörige Quellcode ist im Anhang dokumentiert. Ein veranschaulichter Gratverlauf ist im folgendem Kapitel ~\ref{sec:validierung-burr} in Abbildung ~\ref{fig:burr-compare} dargestellt.

% \section{Rauheitsanalyse bei Edelstahl}
% \label{sec:roughness-analysis}

% Für die Rauheitsanalyse wird die bestehende Auswertelogik aus der Baustahl-Pipeline übernommen und auf Edelstahl angewendet. Ziel ist die Überprüfung, ob die Parameter und Rechenschritte ohne Anpassungen auch für Edelstahl robuste und nachvollziehbare Werte liefern. Dabei liegt der Fokus auf der Erkennung unplausibler Ergebnisse, die auf ungeeignete Aufnahmebedingungen oder auf notwendigen Anpassungsbedarf im Skript hinweisen könnten.

% Die Bildaufnahmeeinstellungen des Handscanners wurden zuvor für Edelstahl optimiert, um Effekte durch erhöhte Reflexion zu minimieren, siehe \cref{chap:handscanner-setup}. Auftretende Unplausibilitäten wären damit primär dem Auswertungsskript zuzuordnen. Entsprechend wurde die Rauheitsanalyse anhand einer Stichprobe validiert. Untersucht wurden zehn Edelstahlbleche mit jeweils vier Schnittkanten. Die Prüfung umfasste Konsistenztests über die gesamte Kantenlänge, die Kontrolle auf Ausreißer sowie die Sichtprüfung auf Sättigungseffekte und Messlücken. Keine der vierzig untersuchten Kanten zeigte auffällig abweichende Rauheitswerte. Das Skript zur Rauheitsanalyse wird daher unverändert beibehalten.

% Die Auswertung erfolgt abschnittsweise entlang der Schnittkante. Die Kante wird von oben nach unten in Segmente der Länge \(0{,}3\,\mathrm{mm}\) unterteilt. Die X-Achse gibt die Position entlang der Schnittkante \(x\,[\mathrm{mm}]\) an, die Y-Achse den für eine Segmentlänge von \(0{,}3\,\mathrm{mm}\) berechneten Rauheitswert \(R_a\,[\mu\mathrm{m}]\).Die Skalierung der x-Achse auf den Bereich \(2{,}1\,\mathrm{cm}\) bis \(7{,}6\,\mathrm{cm}\) ergibt sich aus der Blechbreite von \(10\,\mathrm{cm}\). Ausgewertet wird bewusst nur der mittlere Abschnitt, um beschleunigungsbedingte Einflüsse des Laserschneidprozesses an Ein- und Ausfahrt des Schneidkopfes zu vermeiden. Für jedes Segment wird die Rauheitskennzahl \(R_a\) berechnet und als Profil über der Kantenlänge dargestellt. Die Abbildung ~\ref{fig:rauheitsanalyse} zeigt ein typisches Ergebnis. Das Profil weist eine geringe Streuung um den Mittelwert auf und enthält keine systematischen Trends oder Ausreißer. Dies bestätigt die Eignung der bestehenden Parametrisierung für Edelstahl.

% \begin{figure}[h]
%   \centering
%   \includegraphics[width=.6\linewidth]{rauheitsanalyse.png}%
%   \caption{Beispielhafter Verlauf der Rauheitskennzahl \(R_a\) entlang einer Schnittkante im ersten Abschnitt mit der Breite 0,3 mm.}
%   \label{fig:rauheitsanalyse}
% \end{figure}


% \section {Validierung der Messoptimierungen}
% \label{sec:validierung-burr}

% Im folgendem Kapitel werden die verbesserten Messmethoden für die Grat- und Rauheitsanalyse validiert, um bestenfalls die Optimierung für Datenerfassung zu nutzen, so dass das KI-Modell neu trainiert werden kann. 

% \subsection{Validierung der verbesserten Ausreißererkennung im Burr-Verlauf}

% In diesem Abschnitt werden die edelstahl-optimierten Einstellungen der Ausreißererkennung validiert. Verglichen wird die frühere Baustahl-Parametrierung mit der angepassten Erkennung für Edelstahl.

% Die Abbildung~\ref{fig:burr-compare} zeigt den Vergleich der verbsserten und der unverbeserten Ausreißererkennung. Hierfür wurden jeweils das gleiche Sektionsbild des gleichen Edelstahlblechs genutzt. Die rote Linie ist der abgeleitete Burr-Verlauf. Auf dem linken Schaubild ist die unverbesserte Ausreißererkennung erkennbar, welche für Baustahl werdet wird und rechts die verbesserte Ausreißererkennung. Demnach ist der verbesserte Verlauf durchgängig, es entstehen weniger fehlerhafte Lücken und die interpolierten Abschnitte sind plausibler. Quantitativ sinkt der Burr-Wert der gezeigten Sektion von \(681{,}99\,\mu\mathrm{m}\) (Baustahl-Parameter) auf \(650{,}88\,\mu\mathrm{m}\) (Edelstahl-Parameter), jedoch erfolgt die einschätzende Bewertung visuell. Es wird geprüft, wie gut die rote Linie dem sichtbaren Gratverlauf folgt.  
% Demnach ist die verbesserte Ausreißererkennung für Edelstahl validiert.
% \begin{figure}[htbp]
%   \centering
%   \begin{minipage}[t]{0.48\linewidth}
%     \centering
%     \includegraphics[width=\linewidth]{outlier_detection_baustahl.png}
%     \vspace{0.3em}
%     {\small\emph{Unverbesserte Ausreißererkennung: sichtbare Lücken und Fehlverfolgungen.}}
%   \end{minipage}\hfill
%   \begin{minipage}[t]{0.48\linewidth}
%     \centering
%     \includegraphics[width=\linewidth]{outlier_detection_edelstahl.png}
%     \vspace{0.3em}
%     {\small\emph{Verbesserte Ausreißererkennung: durchgängiger Verlauf, plausiblere Interpolation.}}
%   \end{minipage}
%   \caption[Ausreißererkennung im Burr-Verlauf]{Links ist die unverbesserte Ausreißererkennung erkennbar, welche für Baustahl werdet wird und rechts die verbesserte Ausreißererkennung. Die rote Kurve ist der abgeleitete Burr-Verlauf.}
%   \label{fig:burr-compare}
% \end{figure}

% \subsection{Validierung der bestehenden Rauheitsanalyse an den neuen Edelsdtahldaten}
% In diesem Abschnitt wird die Rauheitsanalyse validiert. Die Berechnungsmethode bleibt unverändert. Die Belichtung des Handscanners wurde so angepasst, dass die Struktur der Schnittfläche zuverlässig erfasst wird (siehe Abschnitt~\ref{chap:handscanner-setup}). Bei der Analyse in Kapitel ~\ref{fig:rauheitsanalyse} traten keine systematischen Ausreißer auf, daher wird die Berechnung vorerst unverändert weitergeführt. Anpassungen werden erst nach weiterführenden Modelltrainings geprüft.

\chapter{Qualitätsschätzungsmodell für Edelstahl}
\label{chap:qa-edelstahl}

Dieses Kapitel beschreibt das bestehende \ac{QA} auf Basis eines \ac{CNNs} und seine Anpassung an Edelstahl. Als Referenz dient VGG16, ein tiefes Netz aus dreizehn Faltungs- und drei vollständig verbundenen Schichten. Kennzeichnend sind gestapelte \(3\times3\)-Faltungen in Blöcken, eine schrittweise Verkleinerung der Feature-Maps und eine mit der Tiefe zunehmende Kanalzahl. VGG16 umfasst rund 138 Millionen Parameter und ist damit leistungsfähig, aber für unsere Datenbasis zu groß. Ein Transfer vortrainierter Gewichte aus ImageNet lieferte auf Schnittkantenbildern keine stabilen, domänerelevanten Merkmale, da sich die dort gelernten Objektstrukturen deutlich von den Texturen und Kanten der Laserschnitte unterscheiden.

Aus den Grundideen von \ac{VGG} wurde daher eine kompaktere, VGG-ähnliche Architektur abgeleitet. Sie behält die Blockstruktur und kleine Faltungskerne bei, reduziert jedoch die Komplexität deutlich und verzichtet auf die breite, mehrstufige Klassifikationskopfkette \parencite{Tatzel-2021}.

Nach den Faltungsblöcken liegen mehrere Merkmalskarten vor.
Global Average Pooling mittelt jede Merkmalskarte über ihre räumlichen Dimensionen und erzeugt so einen kompakten Merkmalsvektor.
Eine schlanke vollständig verbundene Ausgabeschicht bildet diesen Vektor auf die Zielgröße ab, also den geschätzten Gratwert. Das Modell wird von Grund auf auf Schnittkantenbildern trainiert und ist damit auf die materialspezifischen Muster von Edelstahl abgestimmt. \Cref{fig:vgglike-arch} zeigt die Blockstruktur und den kompakten Kopf des Netzes.

\begin{figure}[htbp]
  \centering
  \includegraphics[width=\linewidth]{vgg16_cutting.png}
  \caption{Schematische Darstellung der VGG-ähnlichen Architektur zur Regression der Prozessparameter aus einem Schnittkantenbild. Links der Bildeingang, gefolgt von einem Eingangsblock mit \(7\times7\)-Faltung und vier Blöcken mit jeweils drei \(3\times3\)-Faltungen. Die dritte Faltung eines Blocks reduziert die räumliche Auflösung. Global Average Pooling verdichtet die 256 Feature-Maps zu einem 256-dimensionalen Vektor, der den Ausgange, dem Burr-Wert, speist. \parencite{Tatzel-2021}}
  \label{fig:vgglike-arch}
\end{figure}

\section{Bewertung des bestehenden Modells}

Das bestehende \ac{QA}-Modell für Baustahl wurde auf einen kleinen Edelstahl-Datensatz übertragen und auf die Gratschätzung angewandt. Für die Rauheitsschätzung ist der bestehende Datensatz zu klein, um eine belastbare Bewertung zu ermöglichen. Die Rauhehitsschätzung benötigt eine größere Datenbasis, da die Rauheit stark von lokalen Unregelmäßigkeiten abhängt. Die Gratschätzung ist robuster und kann auch mit weniger Daten bewertet werden.
Abbildung~\ref{fig:burr_true_pred_st} zeigt wahre gegen vorhergesagte Burr-Werte für die Blechdicken 5\,mm, 10\,mm und 15\,mm. Die gestrichelte Diagonale kennzeichnet die ideale Vorhersage $y{=}x$.

Bei kleinen Burr-Werten (hier: 5 mm) liegen die Punkte noch nahe an der Diagonale. Für 10\,mm und 15\,mm werden hohe Burr-Werte deutlich unterschätzt und die Streuung nimmt zu. Ursache ist ein Domänenwechsel von Baustahl zu Edelstahl. Edelstahl zeigt bei größeren Dicken (hier: 10 mm und 15 mm) häufiger hohen Burr. Bei einer Blechdicke von 5 mm ist der Burr-Wert im Verhältnis deutlich geringer. Das Baustahlmodell hat solche Ausprägungen im Training nicht gesehen. Die gelernten Merkmale und die Abbildung von Merkmalen auf den Burr sind dadurch auf den Bereich von Baustahl begrenzt. Das führt zu einer Sättigung der Vorhersage und zu einem dickenabhängigen Bias.

Für eine belastbare Vorhersage auf Edelstahl ist ein erneutes Training mit Edelstahldaten erforderlich. Dabei müssen die Faltungsfilter und nicht nur die letzte Schicht angepasst werden. So können die oberflächenspezifischen Strukturen und die höheren Burr-Werte bei 10\,mm und 15\,mm korrekt erfasst werden.

\begin{figure}[htbp]
  \centering
  \includegraphics[width=0.72\linewidth]{baustahl_modellbewertung.png}
  \caption{Burr: wahre gegen vorhergesagte Werte für das \ac{ST}-Modell. Die Farben kennzeichnen 5\,mm, 10\,mm und 15\,mm. Die gestrichelte Diagonale markiert $y{=}x$.}
  \label{fig:burr_true_pred_st}
\end{figure}

\section{Training für Edelstahldatensatz}

Nach dem Training auf Edelstahldaten für die Gratschätzung zeigt Abbildung~\ref{fig:burr_true_pred_ss} eine enge Übereinstimmung der Punkte mit der Diagonalen.
Der zuvor sichtbare dickenabhängige Bias ist deutlich geringer.
Hohe Burr-Werte bei 10\,mm und 15\,mm werden nun besser getroffen und die Streuung fällt kleiner aus.
Der Grund ist, dass das Modell beim Training nun Beispiele mit hohen Burr-Werten aus Edelstahl gesehen hat.
Diese Ausprägungen waren im Baustahl-Datensatz kaum vorhanden.
Die gelernten Merkmale bilden die für Edelstahl typischen Strukturen daher zuverlässiger ab.
Im höchsten Burr-Bereich bleiben geringe systematische Abweichungen bestehen.
Hier kann eine Erweiterung der Daten mit mehr Fällen hoher Burr-Werte und eine Kalibrierung der Vorhersagen weiter helfen.
Insgesamt liefert das neu trainierte Modell eine robuste und weitgehend dickenunabhängige Gratschätzung für Edelstahl.

\begin{figure}[htbp]
  \centering
  \includegraphics[width=0.72\linewidth]{trainiertes_modell_burr.png}
  \caption{Burr: wahre gegen vorhergesagte Werte für das \ac{SS}-Modell. Die Farben kennzeichnen 5\,mm, 10\,mm und 15\,mm. Die gestrichelte Diagonale markiert $y{=}x$.}
  \label{fig:burr_true_pred_ss}
\end{figure}

\chapter{Reflexion und Ausblick}
\label{chap:reflexion-ausblick}

Dieses Kapitel bewertet die Arbeit und das gewählte Vorgehen rückblickend. Anschließend werden die nächsten Schritte skizziert.

\section{Reflexion}

Das geplante Vorgehen hat sich als zielführend erwiesen. Das Ziel einer verbesserten \ac{QA} für Edelstahlschnittkanten wurde erreicht. Der wesentliche Engpass lag in der Datenerfassung in der Messzelle. Die Messungen pro Parameterkombination benötigen viel Zeit. Für ein stabiles und robustes Modell werden jedoch viele Beispiele benötigt. Deshalb konnte innerhalb des Bearbeitungszeitraums kein Modell für die Rauheit aufgebaut werden.

Die gewählte Reihenfolge war hilfreich. Zuerst wurde die Schnittabrissgrenze analysiert. Darauf basierend wurden Daten nach Versuchsplan erzeugt. Die Messmethodik in der Messzelle wurde auf Edelstahl angepasst und anschließend wurde das Modell auf Edelstahldaten neu trainiert. Dieses Vorgehen ist nachvollziehbar und auf weitere Werkstoffe übertragbar, sofern ausreichend repräsentative Daten vorliegen.

\section{Ausblick}

Künftige Arbeiten sollten prüfen, ob ein gemeinsames Modell für Baustahl und Edelstahl vorteilhaft ist. Ein Ansatz mit gemeinsamem Merkmalsextraktor und materialspezifischen Ausgabeschichten erscheint geeignet. Als Eingaben kommen reine Bilddaten oder eine Kombination aus Bilddaten und Metadaten in Frage. Relevante Metadaten sind insbesondere Material, Blechdicke, Schneidgas und zentrale Prozessparameter.

Für die Rauheitsschätzung ist eine Erweiterung des Datensatzes mit verlässlichen Referenzmessungen erforderlich. Sinnvoll ist ein gemeinsames Training für Grat und Rauheit, da beide Größen aus ähnlichen Strukturen hervorgehen. Eine nachgelagerte Kalibrierung der Vorhersagen erhöht die Anwendbarkeit im Betrieb.

Die Datenerfassung in der Messzelle sollte weiter beschleunigt werden. Automatisierte Messabläufe und aktives Lernen können die Auswahl informativer Parameterkombinationen steuern. Ergänzend helfen Datenaugmentation und eine gezielte Abdeckung hoher Burr-Werte, die bei dickerem Blech häufiger auftreten.

Für eine belastbare Validierung sollten Modelle an unabhängigen Materialchargen und bisher unbekannten Blechdicken getestet werden. Darüber hinaus ist es sinnvoll, Unsicherheiten der Vorhersagen auszugeben und potenzielle Verteilungsunterschiede zwischen Trainings- und Anwendungsdaten zu überwachen. Die Präsentation der Ergebnisse sollte sich an anerkannten Qualitätskennzahlen und relevanten Normen orientieren.

Das hier entwickelte Vorgehen lässt sich auf weitere Legierungen und Blechdicken übertragen. Bei Bedarf wird die Messmethodik angepasst und das Modell mit wenigen neuen Beispielen nachtrainiert, um eine robuste Qualitätsschätzung über unterschiedliche Werkstoffe hinweg zu erreichen.


% \blinddocument
%%% Ende des eigentlichen Inhalts %%%


%%% Quellenverzeichnisse (keine Anpassung nötig) %%%
\clearpage
% \addchap{Anhang}

% \begin{lstlisting}[language={[Sharp]C}, caption={Werkstoffabhängiges Routing der Handscanner-Setups}, label={lst:messzellen-routing}, basicstyle=\scriptsize\ttfamily, breaklines=true, xleftmargin=0.02\textwidth, xrightmargin=0.02\textwidth, numbers=left, numberstyle=\tiny, linewidth=0.96\linewidth, frame=none]
% public static string NameParserMaterial(string input)
% {
%   if (string.IsNullOrWhiteSpace(input))
%     return "ST";                                // Fallback-Wert

%   // Position des ersten und zweiten Bindestrichs ermitteln
%   int firstDash = input.IndexOf('-');
%   int secondDash = firstDash >= 0
%            ? input.IndexOf('-', firstDash + 1)
%            : -1;

%   // Prüfen, ob ein zweiter Bindestrich und ein Zeichen dahinter existieren
%   if (secondDash < 0 || secondDash + 1 >= input.Length)
%     return "ST";                                // Fallback-Wert

%   char digit = input[secondDash + 1];             // Ziffer einlesen

%   // Zuordnung: 1 → "ST", 2 → "ST", 3 → "SS" (bei Bedarf anpassen)
%   return digit switch
%   {
%     '1' => "ST",
%     '2' => "ST",
%     '3' => "SS",
%     _ => "ST"                                   // Default
%   };
% }
% \end{lstlisting}

% \begin{lstlisting}[language=Python, caption={Interpolation begrenzter Lücken in 1D-Profilen}, label={lst:interp-limited-nans}]
% def interpolate_limited_nans(vector, max_gap=5):
%     """
%     Interpoliert nur kleine NaN-Lücken (<= max_gap) mit weicher Spline-Interpolation.
%     """
%     vector = vector.copy()
%     isnan = np.isnan(vector)
%     indices = np.arange(len(vector))

%     if not np.any(~isnan):
%         return None

%     nan_groups = []
%     in_nan = False
%     start = 0
%     for i, val in enumerate(isnan):
%         if val and not in_nan:
%             in_nan = True
%             start = i
%         elif not val and in_nan:
%             in_nan = False
%             nan_groups.append((start, i - 1))
%     if in_nan:
%         nan_groups.append((start, len(vector) - 1))

%     for start, end in nan_groups:
%         gap_size = end - start + 1
%         if gap_size <= max_gap:
%             left = start - 1
%             right = end + 1
%             if left < 0 or right >= len(vector):
%                 continue
%             if np.isnan(vector[left]) or np.isnan(vector[right]):
%                 continue

%             # Verwende CubicSpline statt np.interp
%             x_known = [left, right]
%             y_known = [vector[left], vector[right]]
%             cs = CubicSpline(x_known, y_known, bc_type='natural')
%             interp_indices = indices[start:end+1]
%             vector[start:end+1] = cs(interp_indices)

%     return vector
% \end{lstlisting}

% \begin{lstlisting}[language=Python, caption={Achsweise Interpolation von NaNs in Matrizen}, label={lst:interp-nan-matrix}]
% def interpolate_nan(matrix_in, axis=0, max_gap=5):
%     """
%     Interpoliert NaNs in einer Matrix entlang der gegebenen Achse mit Begrenzung.
%     """
%     matrix = matrix_in.copy()
%     x, y = matrix.shape

%     if axis == 0:
%         for col_index in range(y):
%             if np.isnan(matrix[:, col_index]).any():
%                 vec = interpolate_limited_nans(matrix[:, col_index], max_gap=max_gap)
%                 if vec is not None:
%                     matrix[:, col_index] = vec
%     else:
%         for row_index in range(x):
%             if np.isnan(matrix[row_index, :]).any():
%                 vec = interpolate_limited_nans(matrix[row_index, :], max_gap=max_gap)
%                 if vec is not None:
%                     matrix[row_index, :] = vec
%     return matrix
% \end{lstlisting}

% \begin{lstlisting}[language=Python, caption={Interpolation von X-, Y- und Z-Ebenen}, label={lst:interp-planes}]
% def interpolate_planes(X, Y, Z, results_directory=None, max_gap=0.5):
%     """
%     Interpoliert X, Y, Z Ebenen entlang sinnvoller Achsen mit Lückenbegrenzung.
%     """
%     X_interpolated = interpolate_nan(X, axis=1, max_gap=max_gap)
%     Y_interpolated = interpolate_nan(Y, axis=1, max_gap=max_gap)
%     Z_interpolated = interpolate_nan(Z, axis=0, max_gap=max_gap)
%     return X_interpolated, Y_interpolated, Z_interpolated
% \end{lstlisting}

% \begin{lstlisting}[language=Python, caption={Kapselnde 1D-Pipeline: begrenzte Interpolation mit optionaler Glättung}, label={lst:smooth-nan-values}]
% def smooth_nan_values(x, z, max_nan_values_perc=0.4, max_gap=5, 
%                       smoothing=True, window_length=7, polyorder=2):
%     """
%     Interpoliert NaN-Werte in z, aber nur bei kleiner Lückenanzahl
%     und akzeptablem NaN-Anteil. Optional geglättet mit Savitzky-Golay.
%     """
%     z = np.copy(z)
%     nan_indices = np.isnan(z)
%     num_nans = np.sum(nan_indices)

%     if num_nans / len(z) > max_nan_values_perc:
%         return None

%     z = interpolate_limited_nans(z, max_gap=max_gap)

%     # Optional glätten
%     if smoothing and z is not None and np.count_nonzero(~np.isnan(z)) > window_length:
%         z = savgol_filter(z, window_length=window_length, polyorder=polyorder)

%     return z
% \end{lstlisting}

% \begin{lstlisting}[language=Python, caption={Schließen kurzer Täler (Valleys) via begrenzter Interpolation}, label={lst:fill-small-valleys}]
% def fill_small_valleys(z, window_size=30, depth_threshold=5.0, max_gap=0.5,
%                        smoothing=True, smooth_window=7, polyorder=2):
%     """
%     Glättet kleine Einbrüche ('valleys') in einem Höhenprofil z.
%     """
%     z = np.array(z).copy()
%     rolling_min = pd.Series(z).rolling(window=window_size, center=True, min_periods=5).min()
%     diff = rolling_min - z

%     # Punkte mit tieferem Einbruch als erlaubt
%     valley_mask = (diff > depth_threshold)
%     z[valley_mask] = np.nan

%     z_filled = interpolate_limited_nans(z, max_gap=max_gap)

%     # Optional glätten
%     if smoothing and z_filled is not None and np.count_nonzero(~np.isnan(z_filled)) > smooth_window:
%         z_filled = savgol_filter(z_filled, window_length=smooth_window, polyorder=2)

%     return z_filled
% \end{lstlisting}

\literaturverzeichnis
%%% Ende Quellenverzeichnisse %%%


%%% Erklärung (keine Anpassungen nötig) %%%
% steht ganz am Ende des Dokuments
\cleardoublepage
%\input{includes/erklaerung_ki.tex}
%\input{template/_dhbw_erklaerung.tex}
\end{document}
